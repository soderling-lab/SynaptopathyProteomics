\documentclass[]{article}
\usepackage{lmodern}
\usepackage{amssymb,amsmath}
\usepackage{ifxetex,ifluatex}
\usepackage{fixltx2e} % provides \textsubscript
\ifnum 0\ifxetex 1\fi\ifluatex 1\fi=0 % if pdftex
  \usepackage[T1]{fontenc}
  \usepackage[utf8]{inputenc}
\else % if luatex or xelatex
  \ifxetex
    \usepackage{mathspec}
  \else
    \usepackage{fontspec}
  \fi
  \defaultfontfeatures{Ligatures=TeX,Scale=MatchLowercase}
\fi
% use upquote if available, for straight quotes in verbatim environments
\IfFileExists{upquote.sty}{\usepackage{upquote}}{}
% use microtype if available
\IfFileExists{microtype.sty}{%
\usepackage{microtype}
\UseMicrotypeSet[protrusion]{basicmath} % disable protrusion for tt fonts
}{}
\usepackage[margin=1in]{geometry}
\usepackage{hyperref}
\PassOptionsToPackage{usenames,dvipsnames}{color} % color is loaded by hyperref
\hypersetup{unicode=true,
            pdftitle={TMT Analysis part 1.},
            pdfauthor={Tyler W Bradshaw},
            colorlinks=true,
            linkcolor=Maroon,
            citecolor=Blue,
            urlcolor=blue,
            breaklinks=true}
\urlstyle{same}  % don't use monospace font for urls
\usepackage{color}
\usepackage{fancyvrb}
\newcommand{\VerbBar}{|}
\newcommand{\VERB}{\Verb[commandchars=\\\{\}]}
\DefineVerbatimEnvironment{Highlighting}{Verbatim}{commandchars=\\\{\}}
% Add ',fontsize=\small' for more characters per line
\usepackage{framed}
\definecolor{shadecolor}{RGB}{248,248,248}
\newenvironment{Shaded}{\begin{snugshade}}{\end{snugshade}}
\newcommand{\KeywordTok}[1]{\textcolor[rgb]{0.13,0.29,0.53}{\textbf{#1}}}
\newcommand{\DataTypeTok}[1]{\textcolor[rgb]{0.13,0.29,0.53}{#1}}
\newcommand{\DecValTok}[1]{\textcolor[rgb]{0.00,0.00,0.81}{#1}}
\newcommand{\BaseNTok}[1]{\textcolor[rgb]{0.00,0.00,0.81}{#1}}
\newcommand{\FloatTok}[1]{\textcolor[rgb]{0.00,0.00,0.81}{#1}}
\newcommand{\ConstantTok}[1]{\textcolor[rgb]{0.00,0.00,0.00}{#1}}
\newcommand{\CharTok}[1]{\textcolor[rgb]{0.31,0.60,0.02}{#1}}
\newcommand{\SpecialCharTok}[1]{\textcolor[rgb]{0.00,0.00,0.00}{#1}}
\newcommand{\StringTok}[1]{\textcolor[rgb]{0.31,0.60,0.02}{#1}}
\newcommand{\VerbatimStringTok}[1]{\textcolor[rgb]{0.31,0.60,0.02}{#1}}
\newcommand{\SpecialStringTok}[1]{\textcolor[rgb]{0.31,0.60,0.02}{#1}}
\newcommand{\ImportTok}[1]{#1}
\newcommand{\CommentTok}[1]{\textcolor[rgb]{0.56,0.35,0.01}{\textit{#1}}}
\newcommand{\DocumentationTok}[1]{\textcolor[rgb]{0.56,0.35,0.01}{\textbf{\textit{#1}}}}
\newcommand{\AnnotationTok}[1]{\textcolor[rgb]{0.56,0.35,0.01}{\textbf{\textit{#1}}}}
\newcommand{\CommentVarTok}[1]{\textcolor[rgb]{0.56,0.35,0.01}{\textbf{\textit{#1}}}}
\newcommand{\OtherTok}[1]{\textcolor[rgb]{0.56,0.35,0.01}{#1}}
\newcommand{\FunctionTok}[1]{\textcolor[rgb]{0.00,0.00,0.00}{#1}}
\newcommand{\VariableTok}[1]{\textcolor[rgb]{0.00,0.00,0.00}{#1}}
\newcommand{\ControlFlowTok}[1]{\textcolor[rgb]{0.13,0.29,0.53}{\textbf{#1}}}
\newcommand{\OperatorTok}[1]{\textcolor[rgb]{0.81,0.36,0.00}{\textbf{#1}}}
\newcommand{\BuiltInTok}[1]{#1}
\newcommand{\ExtensionTok}[1]{#1}
\newcommand{\PreprocessorTok}[1]{\textcolor[rgb]{0.56,0.35,0.01}{\textit{#1}}}
\newcommand{\AttributeTok}[1]{\textcolor[rgb]{0.77,0.63,0.00}{#1}}
\newcommand{\RegionMarkerTok}[1]{#1}
\newcommand{\InformationTok}[1]{\textcolor[rgb]{0.56,0.35,0.01}{\textbf{\textit{#1}}}}
\newcommand{\WarningTok}[1]{\textcolor[rgb]{0.56,0.35,0.01}{\textbf{\textit{#1}}}}
\newcommand{\AlertTok}[1]{\textcolor[rgb]{0.94,0.16,0.16}{#1}}
\newcommand{\ErrorTok}[1]{\textcolor[rgb]{0.64,0.00,0.00}{\textbf{#1}}}
\newcommand{\NormalTok}[1]{#1}
\usepackage{longtable,booktabs}
\usepackage{graphicx,grffile}
\makeatletter
\def\maxwidth{\ifdim\Gin@nat@width>\linewidth\linewidth\else\Gin@nat@width\fi}
\def\maxheight{\ifdim\Gin@nat@height>\textheight\textheight\else\Gin@nat@height\fi}
\makeatother
% Scale images if necessary, so that they will not overflow the page
% margins by default, and it is still possible to overwrite the defaults
% using explicit options in \includegraphics[width, height, ...]{}
\setkeys{Gin}{width=\maxwidth,height=\maxheight,keepaspectratio}
\IfFileExists{parskip.sty}{%
\usepackage{parskip}
}{% else
\setlength{\parindent}{0pt}
\setlength{\parskip}{6pt plus 2pt minus 1pt}
}
\setlength{\emergencystretch}{3em}  % prevent overfull lines
\providecommand{\tightlist}{%
  \setlength{\itemsep}{0pt}\setlength{\parskip}{0pt}}
\setcounter{secnumdepth}{0}
% Redefines (sub)paragraphs to behave more like sections
\ifx\paragraph\undefined\else
\let\oldparagraph\paragraph
\renewcommand{\paragraph}[1]{\oldparagraph{#1}\mbox{}}
\fi
\ifx\subparagraph\undefined\else
\let\oldsubparagraph\subparagraph
\renewcommand{\subparagraph}[1]{\oldsubparagraph{#1}\mbox{}}
\fi

%%% Use protect on footnotes to avoid problems with footnotes in titles
\let\rmarkdownfootnote\footnote%
\def\footnote{\protect\rmarkdownfootnote}

%%% Change title format to be more compact
\usepackage{titling}

% Create subtitle command for use in maketitle
\newcommand{\subtitle}[1]{
  \posttitle{
    \begin{center}\large#1\end{center}
    }
}

\setlength{\droptitle}{-2em}

  \title{TMT Analysis part 1.}
    \pretitle{\vspace{\droptitle}\centering\huge}
  \posttitle{\par}
    \author{Tyler W Bradshaw}
    \preauthor{\centering\large\emph}
  \postauthor{\par}
      \predate{\centering\large\emph}
  \postdate{\par}
    \date{Tue Apr 30 12:31:05 2019}

\usepackage{float}
\floatplacement{figure}{H}

\begin{document}
\maketitle

{
\hypersetup{linkcolor=black}
\setcounter{tocdepth}{2}
\tableofcontents
}
\subsection{Prepare the workspace.}\label{prepare-the-workspace.}

Prepare the R workspace for the analysis. Load all required packages and
custom functions. Set-up directories for saving output files.

\subsection{Load the raw data and sample info (traits) from
excel.}\label{load-the-raw-data-and-sample-info-traits-from-excel.}

The raw peptide intensity data were exported from ProteomeDiscover (PD)
version 2.2. Note that the default export from PD2.x is a unitless
signal to noise ratio, and it is not recommended to use ths for
quantification.

\begin{Shaded}
\begin{Highlighting}[]
\CommentTok{# Load the data from excel using readxl::read_excel}
\NormalTok{datafile <-}\StringTok{ }\KeywordTok{c}\NormalTok{(}
  \StringTok{"4227_TMT_Cortex_Combined_PD_Peptide_Intensity.xlsx"}\NormalTok{,}
  \StringTok{"4227_TMT_Striatum_Combined_PD_Peptide_Intensity.xlsx"}
\NormalTok{)}

\CommentTok{# Load sample information.}
\NormalTok{samplefile <-}\StringTok{ }\KeywordTok{c}\NormalTok{(}
  \StringTok{"4227_TMT_Cortex_Combined_PD_Protein_Intensity_EBD_traits.csv"}\NormalTok{,}
  \StringTok{"4227_TMT_Striatum_Combined_PD_Protein_Intensity_EBD_traits.csv"}
\NormalTok{)}

\CommentTok{# Load the data from PD and sample info.}
\NormalTok{data_PD <-}\StringTok{ }\KeywordTok{read_excel}\NormalTok{(}\KeywordTok{paste}\NormalTok{(datadir, }\StringTok{"/"}\NormalTok{, datafile[type], }\DataTypeTok{sep =} \StringTok{""}\NormalTok{), }\DecValTok{1}\NormalTok{)}
\NormalTok{sample_info <-}\StringTok{ }\KeywordTok{read.csv}\NormalTok{(}\KeywordTok{paste}\NormalTok{(datadir, }\StringTok{"/"}\NormalTok{, samplefile[type], }\DataTypeTok{sep =} \StringTok{""}\NormalTok{))}

\CommentTok{# Insure traits are in matching order.}
\NormalTok{sample_info <-}\StringTok{ }\NormalTok{sample_info[}\KeywordTok{order}\NormalTok{(sample_info}\OperatorTok{$}\NormalTok{Order), ]}
\end{Highlighting}
\end{Shaded}

\subsection{Cleanup and reorganize the data from
PD.}\label{cleanup-and-reorganize-the-data-from-pd.}

The raw data are cleaned up with the custom function \textbf{cleanPD}.

\begin{Shaded}
\begin{Highlighting}[]
\CommentTok{# Load the data from PD.}
\NormalTok{raw_peptide <-}\StringTok{ }\KeywordTok{cleanPD}\NormalTok{(data_PD, sample_info)}
\end{Highlighting}
\end{Shaded}

\subsection{Examine peptide and protein level identification
overalap.}\label{examine-peptide-and-protein-level-identification-overalap.}

Approximately 40,000 unique peptides cooresponding to
\textasciitilde{}3,000 proteins are quantified across all four
experiments. When comparing the peptides identified in each experiment,
the overlap is only \textasciitilde{}25\%. \textasciitilde{}20\$ of
peptides are identified in all four experiments. This means that in
different experiments, the same protein will likely be quantified by
different peptides. This is a major reason the internal reference
scaling (IRS) normalizaiton approach is employed below.

\begin{Shaded}
\begin{Highlighting}[]
\CommentTok{# Total number of unique peptides:}
\NormalTok{nPeptides <-}\StringTok{ }\KeywordTok{format}\NormalTok{(}\KeywordTok{length}\NormalTok{(}\KeywordTok{unique}\NormalTok{(raw_peptide}\OperatorTok{$}\NormalTok{Sequence)), }\DataTypeTok{big.mark =} \StringTok{","}\NormalTok{)}
\KeywordTok{print}\NormalTok{(}\KeywordTok{paste}\NormalTok{(nPeptides, }\StringTok{" unique peptides identified."}\NormalTok{, }\DataTypeTok{sep =} \StringTok{""}\NormalTok{))}
\end{Highlighting}
\end{Shaded}

\begin{verbatim}
## [1] "42,287 unique peptides identified."
\end{verbatim}

\begin{Shaded}
\begin{Highlighting}[]
\CommentTok{# Total number of unique proteins}
\NormalTok{nProteins <-}\StringTok{ }\KeywordTok{format}\NormalTok{(}\KeywordTok{length}\NormalTok{(}\KeywordTok{unique}\NormalTok{(raw_peptide}\OperatorTok{$}\NormalTok{Accession)), }\DataTypeTok{big.mark =} \StringTok{","}\NormalTok{)}
\KeywordTok{print}\NormalTok{(}\KeywordTok{paste}\NormalTok{(nProteins, }\StringTok{" unique proteins identified."}\NormalTok{,}\DataTypeTok{sep =} \StringTok{""}\NormalTok{))}
\end{Highlighting}
\end{Shaded}

\begin{verbatim}
## [1] "3,492 unique proteins identified."
\end{verbatim}

\begin{Shaded}
\begin{Highlighting}[]
\CommentTok{# Table}
\NormalTok{mytable <-}\StringTok{ }\KeywordTok{data.frame}\NormalTok{(}\DataTypeTok{nPeptides =}\NormalTok{ nPeptides,}
                         \DataTypeTok{nProteins =}\NormalTok{ nProteins)}
\NormalTok{table <-}\StringTok{ }\KeywordTok{tableGrob}\NormalTok{(mytable, }\DataTypeTok{rows =} \OtherTok{NULL}\NormalTok{, }\DataTypeTok{theme =} \KeywordTok{ttheme_default}\NormalTok{())}
\KeywordTok{grid.arrange}\NormalTok{(table)}
\end{Highlighting}
\end{Shaded}

\includegraphics{1_TMT_Analysis_files/figure-latex/unnamed-chunk-12-1.pdf}

\begin{Shaded}
\begin{Highlighting}[]
\CommentTok{# Save as tiff.}
\NormalTok{file <-}\StringTok{ }\KeywordTok{paste0}\NormalTok{(outputfigsdir, }\StringTok{"/"}\NormalTok{, outputMatName, }\StringTok{"_Raw_nPeptides_nProteins.tiff"}\NormalTok{)}
\KeywordTok{ggsave}\NormalTok{(file,table)}
\end{Highlighting}
\end{Shaded}

\begin{verbatim}
## Saving 6.5 x 4.5 in image
\end{verbatim}

\begin{Shaded}
\begin{Highlighting}[]
\CommentTok{# Examine the number of peptides per protein.}
\NormalTok{nPep <-}\StringTok{ }\KeywordTok{subset}\NormalTok{(raw_peptide) }\OperatorTok
\StringTok{  }\KeywordTok{group_by}\NormalTok{(Accession) }\OperatorTok
\StringTok{  }\NormalTok{dplyr}\OperatorTok{::}\KeywordTok{summarize}\NormalTok{(}\DataTypeTok{nPeptides =} \KeywordTok{length}\NormalTok{(Sequence))}

\CommentTok{# Creat table with stats.}
\NormalTok{stats <-}\StringTok{ }\KeywordTok{as.matrix}\NormalTok{(}\KeywordTok{summary}\NormalTok{(nPep}\OperatorTok{$}\NormalTok{nPeptides))}
\NormalTok{df <-}\StringTok{ }\KeywordTok{add_column}\NormalTok{(}\KeywordTok{as.data.frame}\NormalTok{(stats), }\KeywordTok{rownames}\NormalTok{(stats), }\DataTypeTok{.before =} \DecValTok{1}\NormalTok{)}
\NormalTok{tt <-}\StringTok{ }\KeywordTok{ttheme_default}\NormalTok{(}\DataTypeTok{base_size =} \DecValTok{11}\NormalTok{, }\DataTypeTok{core =} \KeywordTok{list}\NormalTok{(}\DataTypeTok{bg_params =} \KeywordTok{list}\NormalTok{(}\DataTypeTok{fill =} \StringTok{"white"}\NormalTok{)))}
\NormalTok{tab <-}\StringTok{ }\KeywordTok{tableGrob}\NormalTok{(df, }\DataTypeTok{cols =} \OtherTok{NULL}\NormalTok{, }\DataTypeTok{rows =} \OtherTok{NULL}\NormalTok{, }\DataTypeTok{theme =}\NormalTok{ tt)}
\NormalTok{g <-}\StringTok{ }\KeywordTok{gtable_add_grob}\NormalTok{(tab,}
  \DataTypeTok{grobs =} \KeywordTok{rectGrob}\NormalTok{(}\DataTypeTok{gp =} \KeywordTok{gpar}\NormalTok{(}\DataTypeTok{fill =} \OtherTok{NA}\NormalTok{, }\DataTypeTok{lwd =} \DecValTok{1}\NormalTok{)),}
  \DataTypeTok{t =} \DecValTok{1}\NormalTok{, }\DataTypeTok{b =} \KeywordTok{nrow}\NormalTok{(tab), }\DataTypeTok{l =} \DecValTok{1}\NormalTok{, }\DataTypeTok{r =} \KeywordTok{ncol}\NormalTok{(tab)}
\NormalTok{)}

\CommentTok{# Generate plot.}
\NormalTok{plot <-}\StringTok{ }\KeywordTok{ggplot}\NormalTok{(nPep, }\KeywordTok{aes}\NormalTok{(nPeptides)) }\OperatorTok{+}\StringTok{ }\KeywordTok{geom_histogram}\NormalTok{(}\DataTypeTok{bins =} \DecValTok{100}\NormalTok{, }\DataTypeTok{fill =} \StringTok{"black"}\NormalTok{) }\OperatorTok{+}
\StringTok{  }\KeywordTok{ggtitle}\NormalTok{(}\StringTok{"Number of peptides per protein"}\NormalTok{) }\OperatorTok{+}
\StringTok{  }\KeywordTok{theme}\NormalTok{(}
    \DataTypeTok{plot.title =} \KeywordTok{element_text}\NormalTok{(}\DataTypeTok{hjust =} \FloatTok{0.5}\NormalTok{, }\DataTypeTok{color =} \StringTok{"black"}\NormalTok{, }\DataTypeTok{size =} \DecValTok{14}\NormalTok{, }\DataTypeTok{face =} \StringTok{"bold"}\NormalTok{),}
    \DataTypeTok{axis.title.x =} \KeywordTok{element_text}\NormalTok{(}\DataTypeTok{color =} \StringTok{"black"}\NormalTok{, }\DataTypeTok{size =} \DecValTok{11}\NormalTok{, }\DataTypeTok{face =} \StringTok{"bold"}\NormalTok{),}
    \DataTypeTok{axis.title.y =} \KeywordTok{element_text}\NormalTok{(}\DataTypeTok{color =} \StringTok{"black"}\NormalTok{, }\DataTypeTok{size =} \DecValTok{11}\NormalTok{, }\DataTypeTok{face =} \StringTok{"bold"}\NormalTok{)}
\NormalTok{  )}

\CommentTok{# Add annotation.}
\NormalTok{p <-}\StringTok{ }\KeywordTok{ggranges}\NormalTok{(plot)}\OperatorTok{$}\NormalTok{TopRight }\CommentTok{# ggranges calculates position of annotation.}
\NormalTok{plot <-}\StringTok{ }\NormalTok{plot }\OperatorTok{+}\StringTok{ }\KeywordTok{annotation_custom}\NormalTok{(g, p}\OperatorTok{$}\NormalTok{xmin, p}\OperatorTok{$}\NormalTok{xmax, p}\OperatorTok{$}\NormalTok{ymin, p}\OperatorTok{$}\NormalTok{ymax)}
\end{Highlighting}
\end{Shaded}

\begin{figure}

{\centering \includegraphics{1_TMT_Analysis_files/figure-latex/unnamed-chunk-13-1} 

}

\caption{Number of peptides identified per protein.}\label{fig:unnamed-chunk-13}
\end{figure}

\begin{verbatim}
## Saving 6.5 x 4.5 in image
\end{verbatim}

\begin{figure}

{\centering \includegraphics{1_TMT_Analysis_files/figure-latex/unnamed-chunk-14-1} 

}

\caption{Peptide identification overlap. All experimental pairwise comparisons.}\label{fig:unnamed-chunk-14}
\end{figure}\begin{figure}

{\centering \includegraphics{1_TMT_Analysis_files/figure-latex/unnamed-chunk-15-1} 

}

\caption{Peptide identification overlap.}\label{fig:unnamed-chunk-15}
\end{figure}

\begin{verbatim}
## Saving 6.5 x 4.5 in image
## Saving 6.5 x 4.5 in image
\end{verbatim}

\subsection{Examine the raw data.}\label{examine-the-raw-data.}

The need for normalization is evident in the raw data. Note that in the
MDS plot, samples cluster by experiment--evidence of a batch effect.

\begin{Shaded}
\begin{Highlighting}[]
\NormalTok{data_in <-}\StringTok{ }\NormalTok{raw_peptide}
\NormalTok{title <-}\StringTok{ }\OtherTok{NULL}
\CommentTok{# Colors for boxplot must be specified in ggplot order for boxplot.}
\NormalTok{colors <-}\StringTok{ }\KeywordTok{c}\NormalTok{(}\KeywordTok{rep}\NormalTok{(}\StringTok{"green"}\NormalTok{, }\DecValTok{11}\NormalTok{), }\KeywordTok{rep}\NormalTok{(}\StringTok{"purple"}\NormalTok{, }\DecValTok{11}\NormalTok{), }\KeywordTok{rep}\NormalTok{(}\StringTok{"yellow"}\NormalTok{, }\DecValTok{11}\NormalTok{), }\KeywordTok{rep}\NormalTok{(}\StringTok{"blue"}\NormalTok{, }\DecValTok{11}\NormalTok{))}

\CommentTok{# Generate boxplot.}
\NormalTok{p1 <-}\StringTok{ }\KeywordTok{ggplotBoxPlot}\NormalTok{(data_in, }\DataTypeTok{colID =} \StringTok{"Abundance"}\NormalTok{, colors, title)}

\CommentTok{# Generate density plot.}
\NormalTok{p2 <-}\StringTok{ }\KeywordTok{ggplotDensity}\NormalTok{(data_in, }\DataTypeTok{colID =} \StringTok{"Abundance"}\NormalTok{, title) }\OperatorTok{+}\StringTok{ }\KeywordTok{theme}\NormalTok{(}\DataTypeTok{legend.position =} \StringTok{"none"}\NormalTok{)}
\CommentTok{# Genotype specific colors must be specified in column order.}
\NormalTok{colors <-}\StringTok{ }\KeywordTok{c}\NormalTok{(}\KeywordTok{rep}\NormalTok{(}\StringTok{"yellow"}\NormalTok{, }\DecValTok{11}\NormalTok{), }\KeywordTok{rep}\NormalTok{(}\StringTok{"blue"}\NormalTok{, }\DecValTok{11}\NormalTok{), }\KeywordTok{rep}\NormalTok{(}\StringTok{"green"}\NormalTok{, }\DecValTok{11}\NormalTok{), }\KeywordTok{rep}\NormalTok{(}\StringTok{"purple"}\NormalTok{, }\DecValTok{11}\NormalTok{))}
\NormalTok{p2 <-}\StringTok{ }\NormalTok{p2 }\OperatorTok{+}\StringTok{ }\KeywordTok{scale_color_manual}\NormalTok{(}\DataTypeTok{values =}\NormalTok{ colors)}

\CommentTok{# Generate meanSd plot.}
\NormalTok{p3 <-}\StringTok{ }\KeywordTok{ggplotMeanSdPlot}\NormalTok{(data_in, }\DataTypeTok{colID =} \StringTok{"Abundance"}\NormalTok{, title, }\DataTypeTok{log =} \OtherTok{TRUE}\NormalTok{)}
\end{Highlighting}
\end{Shaded}

\begin{verbatim}
## Warning: Removed 9 rows containing non-finite values (stat_binhex).
\end{verbatim}

\begin{Shaded}
\begin{Highlighting}[]
\CommentTok{# Generate MDS plot.}
\NormalTok{colors <-}\StringTok{ }\KeywordTok{c}\NormalTok{(}\KeywordTok{rep}\NormalTok{(}\StringTok{"yellow"}\NormalTok{, }\DecValTok{3}\NormalTok{), }\KeywordTok{rep}\NormalTok{(}\StringTok{"blue"}\NormalTok{, }\DecValTok{3}\NormalTok{), }\KeywordTok{rep}\NormalTok{(}\StringTok{"green"}\NormalTok{, }\DecValTok{3}\NormalTok{), }\KeywordTok{rep}\NormalTok{(}\StringTok{"purple"}\NormalTok{, }\DecValTok{3}\NormalTok{))}
\NormalTok{p4 <-}\StringTok{ }\KeywordTok{ggplotMDS}\NormalTok{(data_in, }\DataTypeTok{colID =} \StringTok{"Abundance"}\NormalTok{, colors, title, sample_info, }\DataTypeTok{labels =} \OtherTok{TRUE}\NormalTok{) }\OperatorTok{+}
\StringTok{  }\KeywordTok{theme}\NormalTok{(}\DataTypeTok{legend.position =} \StringTok{"none"}\NormalTok{)}

\CommentTok{# Figure.}
\NormalTok{caption <-}\StringTok{ }\KeywordTok{strwrap}\NormalTok{(}\StringTok{"Raw peptide data. A. Boxplot B. Density plot. }
\StringTok{                   C. Mean SD plot. D. MDS plot."}\NormalTok{, }\DataTypeTok{width =} \OtherTok{Inf}\NormalTok{, }\DataTypeTok{simplify =} \OtherTok{TRUE}\NormalTok{)}
\end{Highlighting}
\end{Shaded}

\begin{figure}

{\centering \includegraphics{1_TMT_Analysis_files/figure-latex/unnamed-chunk-19-1} 

}

\caption{Raw peptide data. A. Boxplot B. Density plot.  C. Mean SD plot. D. MDS plot.}\label{fig:unnamed-chunk-19}
\end{figure}

\begin{verbatim}
## Saving 6.5 x 4.5 in image
\end{verbatim}

\subsection{Sample loading normalization within
experiments.}\label{sample-loading-normalization-within-experiments.}

The function \textbf{normalize\_SL} performs sample loading (SL)
normalization to equalize the run level intensity (column) sums. The
data in each column are multiplied by a factor such that the mean of the
column sums are are equal. Sample loading normalization is performed
within an experiment under the assumption that equal amounts of protein
were used for each of the 11 TMT channels.

\begin{Shaded}
\begin{Highlighting}[]
\CommentTok{# Define data columns for SL within experiments:}
\NormalTok{colID <-}\StringTok{ "Abundance"}
\NormalTok{groups <-}\StringTok{ }\KeywordTok{c}\NormalTok{(}\StringTok{"Shank2"}\NormalTok{, }\StringTok{"Shank3"}\NormalTok{, }\StringTok{"Syngap1"}\NormalTok{, }\StringTok{"Ube3a"}\NormalTok{)}

\CommentTok{# Perform SL normalization.}
\NormalTok{SL_peptide <-}\StringTok{ }\KeywordTok{normalize_SL}\NormalTok{(raw_peptide, colID, groups)}
\end{Highlighting}
\end{Shaded}

\subsection{Examine the SL Data.}\label{examine-the-sl-data.}

Sample loading normalization equalizes the run-level sums within an
11-plex TMT experiment.

\begin{Shaded}
\begin{Highlighting}[]
\NormalTok{data_in <-}\StringTok{ }\NormalTok{SL_peptide}
\NormalTok{title <-}\StringTok{ }\OtherTok{NULL}

\CommentTok{# Generate boxplot.}
\NormalTok{colors <-}\StringTok{ }\KeywordTok{c}\NormalTok{(}\KeywordTok{rep}\NormalTok{(}\StringTok{"green"}\NormalTok{, }\DecValTok{11}\NormalTok{), }\KeywordTok{rep}\NormalTok{(}\StringTok{"purple"}\NormalTok{, }\DecValTok{11}\NormalTok{), }\KeywordTok{rep}\NormalTok{(}\StringTok{"yellow"}\NormalTok{, }\DecValTok{11}\NormalTok{), }\KeywordTok{rep}\NormalTok{(}\StringTok{"blue"}\NormalTok{, }\DecValTok{11}\NormalTok{))}
\NormalTok{p1 <-}\StringTok{ }\KeywordTok{ggplotBoxPlot}\NormalTok{(data_in, }\DataTypeTok{colID =} \StringTok{"Abundance"}\NormalTok{, colors, title)}

\CommentTok{# Generate density plot.}
\NormalTok{p2 <-}\StringTok{ }\KeywordTok{ggplotDensity}\NormalTok{(data_in, }\DataTypeTok{colID =} \StringTok{"Abundance"}\NormalTok{, title) }\OperatorTok{+}\StringTok{ }\KeywordTok{theme}\NormalTok{(}\DataTypeTok{legend.position =} \StringTok{"none"}\NormalTok{)}
\NormalTok{colors <-}\StringTok{ }\KeywordTok{c}\NormalTok{(}\KeywordTok{rep}\NormalTok{(}\StringTok{"yellow"}\NormalTok{, }\DecValTok{11}\NormalTok{), }\KeywordTok{rep}\NormalTok{(}\StringTok{"blue"}\NormalTok{, }\DecValTok{11}\NormalTok{), }\KeywordTok{rep}\NormalTok{(}\StringTok{"green"}\NormalTok{, }\DecValTok{11}\NormalTok{), }\KeywordTok{rep}\NormalTok{(}\StringTok{"purple"}\NormalTok{, }\DecValTok{11}\NormalTok{))}
\NormalTok{p2 <-}\StringTok{ }\NormalTok{p2 }\OperatorTok{+}\StringTok{ }\KeywordTok{scale_color_manual}\NormalTok{(}\DataTypeTok{values =}\NormalTok{ colors)}

\CommentTok{# Generate meanSd plot.}
\NormalTok{p3 <-}\StringTok{ }\KeywordTok{ggplotMeanSdPlot}\NormalTok{(data_in, }\DataTypeTok{colID =} \StringTok{"Abundance"}\NormalTok{, title, }\DataTypeTok{log =} \OtherTok{TRUE}\NormalTok{)}
\end{Highlighting}
\end{Shaded}

\begin{verbatim}
## Warning: Removed 9 rows containing non-finite values (stat_binhex).
\end{verbatim}

\begin{Shaded}
\begin{Highlighting}[]
\CommentTok{# Generate MDS plot.}
\NormalTok{colors <-}\StringTok{ }\KeywordTok{c}\NormalTok{(}\KeywordTok{rep}\NormalTok{(}\StringTok{"yellow"}\NormalTok{, }\DecValTok{3}\NormalTok{), }\KeywordTok{rep}\NormalTok{(}\StringTok{"blue"}\NormalTok{, }\DecValTok{3}\NormalTok{), }\KeywordTok{rep}\NormalTok{(}\StringTok{"green"}\NormalTok{, }\DecValTok{3}\NormalTok{), }\KeywordTok{rep}\NormalTok{(}\StringTok{"purple"}\NormalTok{, }\DecValTok{3}\NormalTok{))}
\NormalTok{p4 <-}\StringTok{ }\KeywordTok{ggplotMDS}\NormalTok{(data_in, }\DataTypeTok{colID =} \StringTok{"Abundance"}\NormalTok{, colors, title, sample_info, }\DataTypeTok{labels =} \OtherTok{TRUE}\NormalTok{) }\OperatorTok{+}
\StringTok{  }\KeywordTok{theme}\NormalTok{(}\DataTypeTok{legend.position =} \StringTok{"none"}\NormalTok{)}

\CommentTok{# Figure.}
\NormalTok{caption <-}\StringTok{ }\KeywordTok{strwrap}\NormalTok{(}\StringTok{"Sample loading normalized peptide data. A. Boxplot B. Density plot. }
\StringTok{                   C. Mean SD plot. D. MDS plot."}\NormalTok{, }\DataTypeTok{width =} \OtherTok{Inf}\NormalTok{, }\DataTypeTok{simplify =} \OtherTok{TRUE}\NormalTok{)}
\end{Highlighting}
\end{Shaded}

\begin{figure}

{\centering \includegraphics{1_TMT_Analysis_files/figure-latex/unnamed-chunk-26-1} 

}

\caption{Sample loading normalized peptide data. A. Boxplot B. Density plot.  C. Mean SD plot. D. MDS plot.}\label{fig:unnamed-chunk-26}
\end{figure}

\begin{verbatim}
## Saving 6.5 x 4.5 in image
\end{verbatim}

\subsection{Examine the nature of missing
values.}\label{examine-the-nature-of-missing-values.}

Missing values are inherent in high throughput experiments. There are
two main classes of missing values, missing at random (MAR) and missing
not at random (MNAR). The appropriate imputing algorithm should be
chosen based on the nature of missing values. The distribution of
peptides with missing values is examined by a density plot. The
left-shifted distribution of peptides with missing values indicates that
peptides that have missing values are generally lower in abundance.
Missing values are likely then to be not missing at random (MNAR), but
missing because they are low-abundance and at or near the limit of
detection. MNAR data can be imputed with the k-nearest neighbors (knn)
algorithm in the next chunk.

\begin{Shaded}
\begin{Highlighting}[]
\CommentTok{# Define groups for subseting the data.}
\NormalTok{groups <-}\StringTok{ }\KeywordTok{c}\NormalTok{(}\StringTok{"Shank2"}\NormalTok{, }\StringTok{"Shank3"}\NormalTok{, }\StringTok{"Syngap1"}\NormalTok{, }\StringTok{"Ube3a"}\NormalTok{)}

\CommentTok{# Generate plots.}
\NormalTok{p1 <-}\StringTok{ }\KeywordTok{ggplotDetect}\NormalTok{(SL_peptide, groups[}\DecValTok{1}\NormalTok{]) }\OperatorTok{+}\StringTok{ }\KeywordTok{ggtitle}\NormalTok{(}\OtherTok{NULL}\NormalTok{)}
\NormalTok{p2 <-}\StringTok{ }\KeywordTok{ggplotDetect}\NormalTok{(SL_peptide, groups[}\DecValTok{2}\NormalTok{]) }\OperatorTok{+}\StringTok{ }\KeywordTok{ggtitle}\NormalTok{(}\OtherTok{NULL}\NormalTok{)}
\NormalTok{p3 <-}\StringTok{ }\KeywordTok{ggplotDetect}\NormalTok{(SL_peptide, groups[}\DecValTok{3}\NormalTok{]) }\OperatorTok{+}\StringTok{ }\KeywordTok{ggtitle}\NormalTok{(}\OtherTok{NULL}\NormalTok{)}
\NormalTok{p4 <-}\StringTok{ }\KeywordTok{ggplotDetect}\NormalTok{(SL_peptide, groups[}\DecValTok{4}\NormalTok{]) }\OperatorTok{+}\StringTok{ }\KeywordTok{ggtitle}\NormalTok{(}\OtherTok{NULL}\NormalTok{)}

\CommentTok{# Figure.}
\NormalTok{caption <-}\StringTok{ }\KeywordTok{strwrap}\NormalTok{(}\StringTok{"Peptide-level missing value distributions. A. Shank2 B. Shank3. }
\StringTok{                   C. Syngap1. D. Ube3a."}\NormalTok{, }\DataTypeTok{width =} \OtherTok{Inf}\NormalTok{, }\DataTypeTok{simplify =} \OtherTok{TRUE}\NormalTok{)}
\end{Highlighting}
\end{Shaded}

\begin{figure}

{\centering \includegraphics{1_TMT_Analysis_files/figure-latex/unnamed-chunk-30-1} 

}

\caption{Peptide-level missing value distributions. A. Shank2 B. Shank3.  C. Syngap1. D. Ube3a.}\label{fig:unnamed-chunk-30}
\end{figure}

\begin{verbatim}
## Saving 6.5 x 4.5 in image
\end{verbatim}

\subsection{Impute missing peptide values within an
experiment.}\label{impute-missing-peptide-values-within-an-experiment.}

The function \textbf{impute\_Peptides} supports imputing missing values
with the maximum likelyhood estimation (MLE) or KNN algorithms for
missing not at random (MNAR) and missing at random data, respectively.
Impution is performed with an experiment, and rows with more than 50\%
missing values are censored and will not be imputed. Peptides with more
than 2 missing biological replicates or any missing quality control (QC)
replicates will be censored and are not imputed.

\begin{Shaded}
\begin{Highlighting}[]
\CommentTok{# Define experimental groups for checking QC variability:}
\NormalTok{groups <-}\StringTok{ }\KeywordTok{c}\NormalTok{(}\StringTok{"Shank2"}\NormalTok{, }\StringTok{"Shank3"}\NormalTok{, }\StringTok{"Syngap1"}\NormalTok{, }\StringTok{"Ube3a"}\NormalTok{)}

\CommentTok{# Impute missing values using KNN algorithm for MNAR data.}
\CommentTok{# Rows with missing QC replicates are ingored (qc_threshold=0).}
\CommentTok{# Rows with more than 2 (50%) missing biological replicates are }
\CommentTok{# ignored (bio_threshold=2).}
\NormalTok{data_impute <-}\StringTok{ }\KeywordTok{impute_Peptides}\NormalTok{(SL_peptide, groups, }\DataTypeTok{method =} \StringTok{"knn"}\NormalTok{)}
\end{Highlighting}
\end{Shaded}

\begin{verbatim}
## 128 values from Shank2 are missing and will be replaced by imputing.
## 57 values from Shank3 are missing and will be replaced by imputing.
## 75 values from Syngap1 are missing and will be replaced by imputing.
## 89 values from Ube3a are missing and will be replaced by imputing.
\end{verbatim}

\begin{Shaded}
\begin{Highlighting}[]
\NormalTok{impute_peptide <-}\StringTok{ }\NormalTok{data_impute}\OperatorTok{$}\NormalTok{data_imputed}

\CommentTok{# Table of n imputed peptides.}
\NormalTok{n_out <-}\StringTok{ }\NormalTok{data_impute}\OperatorTok{$}\NormalTok{n_out}
\NormalTok{table <-}\StringTok{ }\KeywordTok{as.data.frame}\NormalTok{(}\KeywordTok{do.call}\NormalTok{(rbind, n_out))}
\NormalTok{table <-}\StringTok{ }\KeywordTok{add_column}\NormalTok{(table, }\KeywordTok{rownames}\NormalTok{(table), }\DataTypeTok{.before =} \DecValTok{1}\NormalTok{)}
\KeywordTok{colnames}\NormalTok{(table) <-}\StringTok{ }\KeywordTok{c}\NormalTok{(}\StringTok{"Experiment"}\NormalTok{, }\StringTok{"N Imputed"}\NormalTok{)}
\NormalTok{table <-}\StringTok{ }\KeywordTok{tableGrob}\NormalTok{(table, }\DataTypeTok{rows =} \OtherTok{NULL}\NormalTok{, }\DataTypeTok{theme =} \KeywordTok{ttheme_default}\NormalTok{())}

\CommentTok{# Table.}
\end{Highlighting}
\end{Shaded}

\begin{figure}

{\centering \includegraphics{1_TMT_Analysis_files/figure-latex/unnamed-chunk-34-1} 

}

\caption{Summary of imputed values.}\label{fig:unnamed-chunk-34}
\end{figure}

\begin{verbatim}
## Saving 6.5 x 4.5 in image
\end{verbatim}

\subsection{Illustrate the mean variance relationship of QC
peptides.}\label{illustrate-the-mean-variance-relationship-of-qc-peptides.}

Quality control samples can be used to asses intra-experimental
variability. Peptides that have highly variable QC measurements will
increase protein level variability and should be removed. The
peptide-level QC data are binned by intensity, and peptides whose mean
ratio are more than four standard deviations away from the mean are
considered outliers and removed.

\begin{Shaded}
\begin{Highlighting}[]
\CommentTok{# Generate QC correlation scatter plots for all experimental groups.}
\NormalTok{groups <-}\StringTok{ }\KeywordTok{c}\NormalTok{(}\StringTok{"Shank2"}\NormalTok{, }\StringTok{"Shank3"}\NormalTok{, }\StringTok{"Syngap1"}\NormalTok{, }\StringTok{"Ube3a"}\NormalTok{)}
\NormalTok{plots <-}\StringTok{ }\KeywordTok{ggplotcorQC}\NormalTok{(impute_peptide, groups, }\DataTypeTok{colID =} \StringTok{"QC"}\NormalTok{, }\DataTypeTok{nbins =} \DecValTok{5}\NormalTok{)}

\CommentTok{# Generate intensity bin histograms. Example, Shank2.}
\NormalTok{hist_list <-}\StringTok{ }\KeywordTok{list}\NormalTok{()}
\NormalTok{hist_list[[}\StringTok{"Shank2"}\NormalTok{]] <-}\StringTok{ }\KeywordTok{ggplotQCHist}\NormalTok{(impute_peptide, }\StringTok{"Shank2"}\NormalTok{, }\DataTypeTok{nbins =} \DecValTok{5}\NormalTok{, }\DataTypeTok{threshold =} \DecValTok{4}\NormalTok{)}
\NormalTok{hist_list[[}\StringTok{"Shank3"}\NormalTok{]] <-}\StringTok{ }\KeywordTok{ggplotQCHist}\NormalTok{(impute_peptide, }\StringTok{"Shank3"}\NormalTok{, }\DataTypeTok{nbins =} \DecValTok{5}\NormalTok{, }\DataTypeTok{threshold =} \DecValTok{4}\NormalTok{)}
\NormalTok{hist_list[[}\StringTok{"Syngap1"}\NormalTok{]] <-}\StringTok{ }\KeywordTok{ggplotQCHist}\NormalTok{(impute_peptide, }\StringTok{"Syngap1"}\NormalTok{, }\DataTypeTok{nbins =} \DecValTok{5}\NormalTok{, }\DataTypeTok{threshold =} \DecValTok{4}\NormalTok{)}
\NormalTok{hist_list[[}\StringTok{"Ube3a"}\NormalTok{]] <-}\StringTok{ }\KeywordTok{ggplotQCHist}\NormalTok{(impute_peptide, }\StringTok{"Ube3a"}\NormalTok{, }\DataTypeTok{nbins =} \DecValTok{5}\NormalTok{, }\DataTypeTok{threshold =} \DecValTok{4}\NormalTok{)}

\CommentTok{# Figure.}
\CommentTok{#fixme: should optimize size of figure for faster render on pdf. also, annotation layers are not scaled correctly. }
\end{Highlighting}
\end{Shaded}

\begin{figure}

{\centering \includegraphics{1_TMT_Analysis_files/figure-latex/unnamed-chunk-38-1} 

}

\caption{Shank2.}\label{fig:unnamed-chunk-38}
\end{figure}

\begin{verbatim}
## NULL
\end{verbatim}

\begin{verbatim}
## Saving 6.5 x 4.5 in image
## Saving 6.5 x 4.5 in image
## Saving 6.5 x 4.5 in image
## Saving 6.5 x 4.5 in image
## Saving 6.5 x 4.5 in image
## Saving 6.5 x 4.5 in image
## Saving 6.5 x 4.5 in image
## Saving 6.5 x 4.5 in image
## Saving 6.5 x 4.5 in image
\end{verbatim}

\subsection{Peptide level filtering.}\label{peptide-level-filtering.}

Peptides that were not quantified in all three qc replicates are
removed. The data are binned by intensity, and measruments that are 4xSD
from the mean ratio of the intensity bin are considered outliers and
removed.

\begin{Shaded}
\begin{Highlighting}[]
\CommentTok{# Define experimental groups for checking QC variability:}
\NormalTok{groups <-}\StringTok{ }\KeywordTok{c}\NormalTok{(}\StringTok{"Shank2"}\NormalTok{, }\StringTok{"Shank3"}\NormalTok{, }\StringTok{"Syngap1"}\NormalTok{, }\StringTok{"Ube3a"}\NormalTok{)}

\CommentTok{# Filter peptides based on QC precision.}
\NormalTok{filter_peptide <-}\StringTok{ }\KeywordTok{filterQCv2}\NormalTok{(impute_peptide, groups, }\DataTypeTok{nbins =} \DecValTok{5}\NormalTok{, }\DataTypeTok{threshold =} \DecValTok{4}\NormalTok{)}
\end{Highlighting}
\end{Shaded}

\begin{verbatim}
## [1] "182 peptides will be removed from  Shank2  because of QC imprecision"
## [1] "67 peptides will be removed from  Shank3  because of QC imprecision"
## [1] "73 peptides will be removed from  Syngap1  because of QC imprecision"
## [1] "75 peptides will be removed from  Ube3a  because of QC imprecision"
\end{verbatim}

\begin{Shaded}
\begin{Highlighting}[]
\CommentTok{# Generate table.}
\NormalTok{out <-}\StringTok{ }\KeywordTok{list}\NormalTok{(}\KeywordTok{c}\NormalTok{(}\DecValTok{94}\NormalTok{,}\DecValTok{77}\NormalTok{,}\DecValTok{133}\NormalTok{,}\DecValTok{59}\NormalTok{),}\KeywordTok{c}\NormalTok{(}\DecValTok{182}\NormalTok{,}\DecValTok{67}\NormalTok{,}\DecValTok{73}\NormalTok{,}\DecValTok{75}\NormalTok{))[[type]] }\CommentTok{# Cox and Str peps removed.}
\NormalTok{mytable <-}\StringTok{ }\KeywordTok{data.frame}\NormalTok{(}\KeywordTok{cbind}\NormalTok{(groups,out))}
\NormalTok{table <-}\StringTok{ }\KeywordTok{tableGrob}\NormalTok{(mytable, }\DataTypeTok{rows =} \OtherTok{NULL}\NormalTok{, }\DataTypeTok{theme =} \KeywordTok{ttheme_default}\NormalTok{())}
\KeywordTok{grid.arrange}\NormalTok{(table)}
\end{Highlighting}
\end{Shaded}

\includegraphics{1_TMT_Analysis_files/figure-latex/unnamed-chunk-41-1.pdf}

\begin{Shaded}
\begin{Highlighting}[]
\CommentTok{# Save as tiff.}
\NormalTok{file <-}\StringTok{ }\KeywordTok{paste0}\NormalTok{(outputfigsdir, }\StringTok{"/"}\NormalTok{, outputMatName, }\StringTok{"_nProteins_Filtered.tiff"}\NormalTok{)}
\KeywordTok{ggsave}\NormalTok{(file,table)}
\end{Highlighting}
\end{Shaded}

\begin{verbatim}
## Saving 6.5 x 4.5 in image
\end{verbatim}

\subsection{Protein level summarization and normalization across all
batches.}\label{protein-level-summarization-and-normalization-across-all-batches.}

Summarize to protein level by summing peptide intensities. Note that the
peptide column in the returned data frame reflects the total number of
peptides identified for a given protein across all experiments.

\begin{Shaded}
\begin{Highlighting}[]
\CommentTok{# Summarize to protein level:}
\NormalTok{SL_protein <-}\StringTok{ }\KeywordTok{summarize_Protein}\NormalTok{(filter_peptide)}

\CommentTok{# Normalize across all columns (experiments).}
\NormalTok{SL_protein <-}\StringTok{ }\KeywordTok{normalize_SL}\NormalTok{(SL_protein, }\StringTok{"Abundance"}\NormalTok{, }\StringTok{"Abundance"}\NormalTok{)}
\end{Highlighting}
\end{Shaded}

\subsection{IntraBatch Protein-lavel
ComBat.}\label{intrabatch-protein-lavel-combat.}

Each experimental cohort of 8 was prepared in two batches. This was
necessary because the ultra-centrifuge holds a maximum of 6 samples.
This intra-batch batch effect was recorded for 6/8 experiments. We will
utilize the \textbf{ComBat()} function from the sva package to remove
this batch effect before attempting to correct the inter-batch effect
between batches with IRS normalization.

\begin{Shaded}
\begin{Highlighting}[]
\CommentTok{# Define experimental groups and column ID for expression data.}
\NormalTok{groups}
\end{Highlighting}
\end{Shaded}

\begin{verbatim}
## [1] "Shank2"  "Shank3"  "Syngap1" "Ube3a"
\end{verbatim}

\begin{Shaded}
\begin{Highlighting}[]
\NormalTok{colID <-}\StringTok{ "Abundance"}
\NormalTok{data_in <-}\StringTok{ }\NormalTok{SL_protein}

\CommentTok{# Loop to perform ComBat on intraBatch batch effect (prep date).}
\CommentTok{# If there is no known batch effect, the data is returned un-regressed.}
\CommentTok{# Note: QC samples are not adjusted by ComBat.}

\NormalTok{data_out <-}\StringTok{ }\KeywordTok{list}\NormalTok{() }\CommentTok{# ComBat data.}
\NormalTok{plot_list <-}\StringTok{ }\KeywordTok{list}\NormalTok{() }\CommentTok{# MDS plots.}
\NormalTok{R <-}\StringTok{ }\KeywordTok{list}\NormalTok{() }\CommentTok{# Bicor stats [bicor(batch,PC1)]}

\ControlFlowTok{for}\NormalTok{ (i }\ControlFlowTok{in} \DecValTok{1}\OperatorTok{:}\KeywordTok{length}\NormalTok{(groups)) \{}
  \CommentTok{# Meta data.}
\NormalTok{  info_cols <-}\StringTok{ }\NormalTok{data_in[, }\OperatorTok{!}\KeywordTok{grepl}\NormalTok{(colID, }\KeywordTok{colnames}\NormalTok{(data_in))]}

  \CommentTok{# Expression data.}
\NormalTok{  group <-}\StringTok{ }\NormalTok{groups[[i]]}
\NormalTok{  cols <-}\StringTok{ }\KeywordTok{grepl}\NormalTok{(group, }\KeywordTok{colnames}\NormalTok{(data_in))}
\NormalTok{  data_work <-}\StringTok{ }\KeywordTok{as.matrix}\NormalTok{(data_in[, cols])}
  \KeywordTok{rownames}\NormalTok{(data_work) <-}\StringTok{ }\KeywordTok{paste}\NormalTok{(data_in}\OperatorTok{$}\NormalTok{Accession,}
    \KeywordTok{c}\NormalTok{(}\DecValTok{1}\OperatorTok{:}\KeywordTok{nrow}\NormalTok{(data_in)),}
    \DataTypeTok{sep =} \StringTok{"_"}
\NormalTok{  )}
\NormalTok{  rows_out <-}\StringTok{ }\KeywordTok{apply}\NormalTok{(data_work, }\DecValTok{1}\NormalTok{, }\ControlFlowTok{function}\NormalTok{(x) }\KeywordTok{sum}\NormalTok{(}\KeywordTok{is.na}\NormalTok{(x) }\OperatorTok{>}\StringTok{ }\DecValTok{0}\NormalTok{))}
\NormalTok{  data <-}\StringTok{ }\NormalTok{data_work[}\OperatorTok{!}\NormalTok{rows_out, ]}

  \CommentTok{# Get Traits info.}
\NormalTok{  idx <-}\StringTok{ }\KeywordTok{match}\NormalTok{(}\KeywordTok{colnames}\NormalTok{(data), sample_info}\OperatorTok{$}\NormalTok{ColumnName)}
\NormalTok{  traits_sub <-}\StringTok{ }\NormalTok{sample_info[idx, ]}
  \KeywordTok{rownames}\NormalTok{(traits_sub) <-}\StringTok{ }\NormalTok{traits_sub}\OperatorTok{$}\NormalTok{ColumnName}

  \CommentTok{# QC Samples will be ignored.}
\NormalTok{  ignore <-}\StringTok{ }\KeywordTok{is.na}\NormalTok{(traits_sub}\OperatorTok{$}\NormalTok{PrepDate)}
\NormalTok{  data_QC <-}\StringTok{ }\NormalTok{data[, ignore]}
\NormalTok{  CombatInfo <-}\StringTok{ }\NormalTok{traits_sub[}\OperatorTok{!}\NormalTok{ignore, ]}
\NormalTok{  data <-}\StringTok{ }\NormalTok{data[, }\OperatorTok{!}\NormalTok{ignore]}

  \CommentTok{# There should be no negative values.}
  \ControlFlowTok{if}\NormalTok{ (}\KeywordTok{min}\NormalTok{(data, }\DataTypeTok{na.rm =} \OtherTok{TRUE}\NormalTok{) }\OperatorTok{<}\StringTok{ }\DecValTok{1}\NormalTok{) \{}
\NormalTok{    data[data }\OperatorTok{<}\StringTok{ }\DecValTok{1}\NormalTok{] <-}\StringTok{ }\DecValTok{1}
    \KeywordTok{print}\NormalTok{(}\StringTok{"Warning: Expression values less than 1 will be replaced with 1."}\NormalTok{)}
\NormalTok{  \}}

  \CommentTok{# Check the correlation between batch and PC1.}
\NormalTok{  pc1 <-}\StringTok{ }\KeywordTok{prcomp}\NormalTok{(}\KeywordTok{t}\NormalTok{(}\KeywordTok{log2}\NormalTok{(data)))}\OperatorTok{$}\NormalTok{x[, }\DecValTok{1}\NormalTok{]}
\NormalTok{  batch <-}\StringTok{ }\KeywordTok{as.numeric}\NormalTok{(}\KeywordTok{as.factor}\NormalTok{(CombatInfo}\OperatorTok{$}\NormalTok{PrepDate)) }\OperatorTok{-}\StringTok{ }\DecValTok{1}
\NormalTok{  r1 <-}\StringTok{ }\KeywordTok{suppressWarnings}\NormalTok{(}\KeywordTok{bicor}\NormalTok{(batch, pc1))}

  \CommentTok{# Check that r2 is not NA.}
  \ControlFlowTok{if}\NormalTok{ (}\KeywordTok{is.na}\NormalTok{(r1)) \{}
\NormalTok{    r1 <-}\StringTok{ }\DecValTok{0}
\NormalTok{  \}}

  \CommentTok{# Check, in matching order?}
  \ControlFlowTok{if}\NormalTok{ (}\OperatorTok{!}\KeywordTok{all}\NormalTok{(}\KeywordTok{colnames}\NormalTok{(data) }\OperatorTok{==}\StringTok{ }\NormalTok{CombatInfo}\OperatorTok{$}\NormalTok{ColumnName)) \{}
    \KeywordTok{print}\NormalTok{(}\StringTok{"Warning: Names of traits and expression data do not match."}\NormalTok{)}
\NormalTok{  \}}

  \CommentTok{# Check MDS plot prior to ComBat.}
\NormalTok{  traits_sub}\OperatorTok{$}\NormalTok{Sample.Model <-}\StringTok{ }\KeywordTok{paste}\NormalTok{(}\StringTok{"b"}\NormalTok{, }
                                   \KeywordTok{as.numeric}\NormalTok{(}\KeywordTok{as.factor}\NormalTok{(traits_sub}\OperatorTok{$}\NormalTok{PrepDate)), }\DataTypeTok{sep =} \StringTok{"."}\NormalTok{)}
\NormalTok{  title <-}\StringTok{ }\KeywordTok{paste}\NormalTok{(}\KeywordTok{gsub}\NormalTok{(}\StringTok{" "}\NormalTok{, }\StringTok{""}\NormalTok{, }\KeywordTok{unique}\NormalTok{(traits_sub}\OperatorTok{$}\NormalTok{Model)), }\StringTok{"pre-ComBat"}\NormalTok{, }\DataTypeTok{sep =} \StringTok{" "}\NormalTok{)}
\NormalTok{  plot1 <-}\StringTok{ }\KeywordTok{ggplotMDSv2}\NormalTok{(}\KeywordTok{log2}\NormalTok{(data),}
    \DataTypeTok{colID =} \StringTok{"b"}\NormalTok{,}
    \DataTypeTok{title =}\NormalTok{ title, }\DataTypeTok{traits =}\NormalTok{ traits_sub}
\NormalTok{  )}\OperatorTok{$}\NormalTok{plot }\OperatorTok{+}\StringTok{ }\KeywordTok{theme}\NormalTok{(}\DataTypeTok{legend.position =} \StringTok{"none"}\NormalTok{)}
\NormalTok{  plot1 <-}\StringTok{ }\NormalTok{plot1 }\OperatorTok{+}\StringTok{ }\KeywordTok{scale_color_manual}\NormalTok{(}\DataTypeTok{values =} \KeywordTok{unique}\NormalTok{(traits_sub}\OperatorTok{$}\NormalTok{Color))}

  \CommentTok{# Apply ComBat.}
  \KeywordTok{cat}\NormalTok{(}\KeywordTok{paste}\NormalTok{(}\StringTok{"Performing"}\NormalTok{, groups[i], }\StringTok{"ComBat..."}\NormalTok{, }\StringTok{"}\CharTok{\textbackslash{}n}\StringTok{"}\NormalTok{))}
  \ControlFlowTok{if}\NormalTok{ (}\KeywordTok{length}\NormalTok{(}\KeywordTok{unique}\NormalTok{(CombatInfo}\OperatorTok{$}\NormalTok{PrepDate)) }\OperatorTok{>}\StringTok{ }\DecValTok{1} \OperatorTok{&}\StringTok{ }\KeywordTok{abs}\NormalTok{(r1) }\OperatorTok{>}\StringTok{ }\FloatTok{0.1}\NormalTok{) \{}
    \CommentTok{# Create ComBat model.}
\NormalTok{    model <-}\StringTok{ }\KeywordTok{model.matrix}\NormalTok{(}\OperatorTok{~}\StringTok{ }\KeywordTok{as.factor}\NormalTok{(CombatInfo}\OperatorTok{$}\NormalTok{SampleType), }
                          \DataTypeTok{data =} \KeywordTok{as.data.frame}\NormalTok{(}\KeywordTok{log2}\NormalTok{(data)))}
\NormalTok{    data_ComBat <-}\StringTok{ }\KeywordTok{ComBat}\NormalTok{(}
      \DataTypeTok{dat =} \KeywordTok{log2}\NormalTok{(data),}
      \DataTypeTok{batch =} \KeywordTok{as.vector}\NormalTok{(CombatInfo}\OperatorTok{$}\NormalTok{PrepDate), }\DataTypeTok{mod =}\NormalTok{ model, }\DataTypeTok{mean.only =} \OtherTok{FALSE}
\NormalTok{    )}
\NormalTok{  \} }\ControlFlowTok{else}\NormalTok{ \{}
    \CommentTok{# No batch effect.}
    \ControlFlowTok{if}\NormalTok{ (}\KeywordTok{abs}\NormalTok{(r1) }\OperatorTok{<}\StringTok{ }\FloatTok{0.1}\NormalTok{) \{}
      \KeywordTok{cat}\NormalTok{(}\KeywordTok{c}\NormalTok{(}
        \StringTok{"Error: No quantifiable batch effect!"}\NormalTok{,}
        \StringTok{"}\CharTok{\textbackslash{}n}\StringTok{"}\NormalTok{, }\StringTok{"The un-regressed data will be returned."}\NormalTok{, }\StringTok{"}\CharTok{\textbackslash{}n}\StringTok{"}
\NormalTok{      ))}
\NormalTok{      data_ComBat <-}\StringTok{ }\KeywordTok{log2}\NormalTok{(data)}
\NormalTok{    \} }\ControlFlowTok{else}\NormalTok{ \{}
      \CommentTok{# No batch effect.}
      \KeywordTok{cat}\NormalTok{(}\KeywordTok{c}\NormalTok{(}
        \StringTok{"Error: ComBat can only be applied to factors with more than two levels!"}\NormalTok{,}
        \StringTok{"}\CharTok{\textbackslash{}n}\StringTok{"}\NormalTok{, }\StringTok{"The un-regressed data will be returned."}\NormalTok{, }\StringTok{"}\CharTok{\textbackslash{}n}\StringTok{"}
\NormalTok{      ))}
\NormalTok{      data_ComBat <-}\StringTok{ }\KeywordTok{log2}\NormalTok{(data)}
\NormalTok{    \}}
\NormalTok{  \}}

  \CommentTok{# Correlation between batch and PC1 post-ComBat.}
\NormalTok{  pc1 <-}\StringTok{ }\KeywordTok{prcomp}\NormalTok{(}\KeywordTok{t}\NormalTok{(data_ComBat))}\OperatorTok{$}\NormalTok{x[, }\DecValTok{1}\NormalTok{]}
\NormalTok{  batch <-}\StringTok{ }\KeywordTok{as.numeric}\NormalTok{(}\KeywordTok{as.factor}\NormalTok{(CombatInfo}\OperatorTok{$}\NormalTok{PrepDate)) }\OperatorTok{-}\StringTok{ }\DecValTok{1}
\NormalTok{  r2 <-}\StringTok{ }\KeywordTok{suppressWarnings}\NormalTok{(}\KeywordTok{bicor}\NormalTok{(batch, pc1))}

  \CommentTok{# Check that r2 is not NA.}
  \ControlFlowTok{if}\NormalTok{ (}\KeywordTok{is.na}\NormalTok{(r2)) \{}
\NormalTok{    r2 <-}\StringTok{ }\DecValTok{0}
\NormalTok{  \}}

\NormalTok{  R[[i]] <-}\StringTok{ }\KeywordTok{cbind}\NormalTok{(r1, r2)}

  \CommentTok{# Check MDS plot after ComBat.}
\NormalTok{  title <-}\StringTok{ }\KeywordTok{paste}\NormalTok{(}\KeywordTok{gsub}\NormalTok{(}\StringTok{" "}\NormalTok{, }\StringTok{""}\NormalTok{, }\KeywordTok{unique}\NormalTok{(traits_sub}\OperatorTok{$}\NormalTok{Model)), }\StringTok{"post-ComBat"}\NormalTok{, }\DataTypeTok{sep =} \StringTok{" "}\NormalTok{)}
\NormalTok{  plot2 <-}\StringTok{ }\KeywordTok{ggplotMDSv2}\NormalTok{(}
    \DataTypeTok{data_in =}\NormalTok{ data_ComBat, }\DataTypeTok{colID =} \StringTok{"Abundance"}\NormalTok{,}
    \DataTypeTok{title =}\NormalTok{ title, }\DataTypeTok{traits =}\NormalTok{ traits_sub}
\NormalTok{  )}\OperatorTok{$}\NormalTok{plot }\OperatorTok{+}\StringTok{ }\KeywordTok{theme}\NormalTok{(}\DataTypeTok{legend.position =} \StringTok{"none"}\NormalTok{)}
\NormalTok{  plot2 <-}\StringTok{ }\NormalTok{plot2 }\OperatorTok{+}\StringTok{ }\KeywordTok{scale_color_manual}\NormalTok{(}\DataTypeTok{values =} \KeywordTok{unique}\NormalTok{(traits_sub}\OperatorTok{$}\NormalTok{Color))}
  \CommentTok{# Add annotation layer.}
  \CommentTok{# xpos <- sum(unlist(ggplot_build(plot2)$layout$panel_params[[1]][1]))/2}
  \CommentTok{# ypos <- sum(unlist(ggplot_build(plot2)$layout$panel_params[[1]][8]))/2}
  \CommentTok{# lab <- paste("bicor(batch,PC1) = ",round(r2$bicor,3))}
  \CommentTok{# plot2 <- plot2 + annotate("text",  x = xpos, y = ypos, label = lab)}

  \CommentTok{# Un-log.}
\NormalTok{  data_ComBat <-}\StringTok{ }\DecValTok{2}\OperatorTok{^}\NormalTok{data_ComBat}
  \CommentTok{# Recombine with QC data.}
\NormalTok{  data_out[[i]] <-}\StringTok{ }\KeywordTok{cbind}\NormalTok{(info_cols[}\OperatorTok{!}\NormalTok{rows_out, ], data_QC, data_ComBat)}
  \KeywordTok{names}\NormalTok{(data_out)[[i]] <-}\StringTok{ }\NormalTok{group}

\NormalTok{  plots <-}\StringTok{ }\KeywordTok{list}\NormalTok{(plot1, plot2)}
\NormalTok{  plot_list[[i]] <-}\StringTok{ }\NormalTok{plots}
  \KeywordTok{names}\NormalTok{(plot_list[[i]]) <-}\StringTok{ }\KeywordTok{paste}\NormalTok{(groups[i], }\KeywordTok{c}\NormalTok{(}\StringTok{"preComBat"}\NormalTok{, }\StringTok{"postComBat"}\NormalTok{))}
\NormalTok{\}}
\end{Highlighting}
\end{Shaded}

\begin{verbatim}
## Performing Shank2 ComBat... 
## Standardizing Data across genes
\end{verbatim}

\begin{verbatim}
## Performing Shank3 ComBat... 
## Error: No quantifiable batch effect! 
##  The un-regressed data will be returned.
\end{verbatim}

\begin{verbatim}
## Performing Syngap1 ComBat... 
## Error: No quantifiable batch effect! 
##  The un-regressed data will be returned.
\end{verbatim}

\begin{verbatim}
## Performing Ube3a ComBat... 
## Standardizing Data across genes
\end{verbatim}

\begin{Shaded}
\begin{Highlighting}[]
\CommentTok{# Merge the data frames with reduce()}
\NormalTok{data_return <-}\StringTok{ }\NormalTok{data_out }\OperatorTok\StringTok{ }\KeywordTok{reduce}\NormalTok{(left_join, }\DataTypeTok{by =} \KeywordTok{c}\NormalTok{(}\KeywordTok{colnames}\NormalTok{(data_in)[}\KeywordTok{c}\NormalTok{(}\DecValTok{1}\NormalTok{, }\DecValTok{2}\NormalTok{)]))}

\CommentTok{# Quantifying the batch effect.}
\CommentTok{# Check bicor correlation with batch before and after ComBat.}
\NormalTok{df <-}\StringTok{ }\KeywordTok{do.call}\NormalTok{(rbind, }\KeywordTok{lapply}\NormalTok{(R, }\ControlFlowTok{function}\NormalTok{(x) x[}\DecValTok{1}\NormalTok{, ]))}
\NormalTok{df <-}\StringTok{ }\KeywordTok{as.data.frame}\NormalTok{(}\KeywordTok{t}\NormalTok{(}\KeywordTok{apply}\NormalTok{(df, }\DecValTok{1}\NormalTok{, }\ControlFlowTok{function}\NormalTok{(x) }\KeywordTok{round}\NormalTok{(}\KeywordTok{abs}\NormalTok{(}\KeywordTok{as.numeric}\NormalTok{(x)), }\DecValTok{3}\NormalTok{))))}
\KeywordTok{rownames}\NormalTok{(df) <-}\StringTok{ }\NormalTok{groups}
\NormalTok{df <-}\StringTok{ }\KeywordTok{add_column}\NormalTok{(df, }\KeywordTok{rownames}\NormalTok{(df), }\DataTypeTok{.before =} \DecValTok{1}\NormalTok{)}
\KeywordTok{colnames}\NormalTok{(df) <-}\StringTok{ }\KeywordTok{c}\NormalTok{(}\StringTok{"Experiment"}\NormalTok{, }\StringTok{"preComBat"}\NormalTok{, }\StringTok{"postComBat"}\NormalTok{)}
\NormalTok{table <-}\StringTok{ }\KeywordTok{tableGrob}\NormalTok{(df, }\DataTypeTok{rows =} \OtherTok{NULL}\NormalTok{)}

\CommentTok{# Table and figures.}
\end{Highlighting}
\end{Shaded}

\begin{figure}

{\centering \includegraphics{1_TMT_Analysis_files/figure-latex/unnamed-chunk-48-1} 

}

\caption{Quantification of the intra-batch batch effect.}\label{fig:unnamed-chunk-48}
\end{figure}

\begin{verbatim}
## Saving 6.5 x 4.5 in image
\end{verbatim}

\begin{figure}

{\centering \includegraphics{1_TMT_Analysis_files/figure-latex/unnamed-chunk-49-1} 

}

\caption{Shank2 ComBat.}\label{fig:unnamed-chunk-49}
\end{figure}\begin{figure}

{\centering \includegraphics{1_TMT_Analysis_files/figure-latex/unnamed-chunk-50-1} 

}

\caption{Shank3 ComBat.}\label{fig:unnamed-chunk-50}
\end{figure}\begin{figure}

{\centering \includegraphics{1_TMT_Analysis_files/figure-latex/unnamed-chunk-51-1} 

}

\caption{Syngap1 ComBat.}\label{fig:unnamed-chunk-51}
\end{figure}\begin{figure}

{\centering \includegraphics{1_TMT_Analysis_files/figure-latex/unnamed-chunk-52-1} 

}

\caption{Ube3a ComBat.}\label{fig:unnamed-chunk-52}
\end{figure}

\subsection{Examine protein identification
overlap.}\label{examine-protein-identification-overlap.}

Approximately 90\% of all proteins are identified in all experiments.

\begin{Shaded}
\begin{Highlighting}[]
\CommentTok{# Inspect the overlap in protein identifcation.}
\NormalTok{plot <-}\StringTok{ }\KeywordTok{ggplotFreqOverlap}\NormalTok{(SL_protein, }\StringTok{"Abundance"}\NormalTok{, groups) }\OperatorTok{+}
\StringTok{  }\KeywordTok{ggtitle}\NormalTok{(}\StringTok{"Protein Identification Overlap"}\NormalTok{)}
\end{Highlighting}
\end{Shaded}

\begin{figure}

{\centering \includegraphics{1_TMT_Analysis_files/figure-latex/unnamed-chunk-56-1} 

}

\caption{Protein identification overlap.}\label{fig:unnamed-chunk-56}
\end{figure}

\begin{verbatim}
## Saving 6.5 x 4.5 in image
\end{verbatim}

\subsection{Examine the Normalized protein level
data.}\label{examine-the-normalized-protein-level-data.}

\begin{Shaded}
\begin{Highlighting}[]
\NormalTok{data_in <-}\StringTok{ }\NormalTok{SL_protein}
\NormalTok{title <-}\StringTok{ "Normalized protein"}

\CommentTok{# Generate boxplot.}
\NormalTok{colors <-}\StringTok{ }\KeywordTok{c}\NormalTok{(}\KeywordTok{rep}\NormalTok{(}\StringTok{"green"}\NormalTok{, }\DecValTok{11}\NormalTok{), }\KeywordTok{rep}\NormalTok{(}\StringTok{"purple"}\NormalTok{, }\DecValTok{11}\NormalTok{), }\KeywordTok{rep}\NormalTok{(}\StringTok{"yellow"}\NormalTok{, }\DecValTok{11}\NormalTok{), }\KeywordTok{rep}\NormalTok{(}\StringTok{"blue"}\NormalTok{, }\DecValTok{11}\NormalTok{))}
\NormalTok{p1 <-}\StringTok{ }\KeywordTok{ggplotBoxPlot}\NormalTok{(data_in, }\DataTypeTok{colID =} \StringTok{"Abundance"}\NormalTok{, colors, title)}

\CommentTok{# Generate density plot.}
\NormalTok{p2 <-}\StringTok{ }\KeywordTok{ggplotDensity}\NormalTok{(data_in, }\DataTypeTok{colID =} \StringTok{"Abundance"}\NormalTok{, title) }\OperatorTok{+}\StringTok{ }\KeywordTok{theme}\NormalTok{(}\DataTypeTok{legend.position =} \StringTok{"none"}\NormalTok{)}
\NormalTok{colors <-}\StringTok{ }\KeywordTok{c}\NormalTok{(}\KeywordTok{rep}\NormalTok{(}\StringTok{"yellow"}\NormalTok{, }\DecValTok{11}\NormalTok{), }\KeywordTok{rep}\NormalTok{(}\StringTok{"blue"}\NormalTok{, }\DecValTok{11}\NormalTok{), }\KeywordTok{rep}\NormalTok{(}\StringTok{"green"}\NormalTok{, }\DecValTok{11}\NormalTok{), }\KeywordTok{rep}\NormalTok{(}\StringTok{"purple"}\NormalTok{, }\DecValTok{11}\NormalTok{))}
\NormalTok{p2 <-}\StringTok{ }\NormalTok{p2 }\OperatorTok{+}\StringTok{ }\KeywordTok{scale_color_manual}\NormalTok{(}\DataTypeTok{values =}\NormalTok{ colors)}

\CommentTok{# Generate meanSd plot.}
\NormalTok{p3 <-}\StringTok{ }\KeywordTok{ggplotMeanSdPlot}\NormalTok{(data_in, }\DataTypeTok{colID =} \StringTok{"Abundance"}\NormalTok{, title, }\DataTypeTok{log =} \OtherTok{TRUE}\NormalTok{)}
\end{Highlighting}
\end{Shaded}

\begin{verbatim}
## Warning: Removed 1 rows containing non-finite values (stat_binhex).
\end{verbatim}

\begin{Shaded}
\begin{Highlighting}[]
\CommentTok{# Generate MDS plot.}
\NormalTok{colors <-}\StringTok{ }\KeywordTok{c}\NormalTok{(}\KeywordTok{rep}\NormalTok{(}\StringTok{"yellow"}\NormalTok{, }\DecValTok{3}\NormalTok{), }\KeywordTok{rep}\NormalTok{(}\StringTok{"blue"}\NormalTok{, }\DecValTok{3}\NormalTok{), }\KeywordTok{rep}\NormalTok{(}\StringTok{"green"}\NormalTok{, }\DecValTok{3}\NormalTok{), }\KeywordTok{rep}\NormalTok{(}\StringTok{"purple"}\NormalTok{, }\DecValTok{3}\NormalTok{))}
\NormalTok{p4 <-}\StringTok{ }\KeywordTok{ggplotMDS}\NormalTok{(data_in, }\DataTypeTok{colID =} \StringTok{"Abundance"}\NormalTok{, colors, title, sample_info, }\DataTypeTok{labels =} \OtherTok{TRUE}\NormalTok{) }\OperatorTok{+}
\StringTok{  }\KeywordTok{theme}\NormalTok{(}\DataTypeTok{legend.position =} \StringTok{"none"}\NormalTok{)}

\CommentTok{# Figure.}
\NormalTok{caption <-}\StringTok{ }\KeywordTok{strwrap}\NormalTok{(}\StringTok{"Normalized protein level data. A. Boxplot. B. Density plot. }
\StringTok{                   C. Mean SD plot. D. MDS plot."}\NormalTok{, }\DataTypeTok{width =} \OtherTok{Inf}\NormalTok{, }\DataTypeTok{simplify =} \OtherTok{TRUE}\NormalTok{)}
\end{Highlighting}
\end{Shaded}

\begin{figure}

{\centering \includegraphics{1_TMT_Analysis_files/figure-latex/unnamed-chunk-60-1} 

}

\caption{Normalized protein level data. A. Boxplot. B. Density plot.  C. Mean SD plot. D. MDS plot.}\label{fig:unnamed-chunk-60}
\end{figure}

\begin{verbatim}
## Saving 6.5 x 4.5 in image
\end{verbatim}

\subsection{IRS Normalization.}\label{irs-normalization.}

Internal reference sclaing (IRS) normalization equalizes the means of
reference (QC) samples across all batches. IRS normalization accounts
for the random sampling of peptides at the MS2 level--proteins are
identified by different peptides in different experiments.

\begin{Shaded}
\begin{Highlighting}[]
\CommentTok{# Perform IRS normaliztion.}
\NormalTok{groups <-}\StringTok{ }\KeywordTok{c}\NormalTok{(}\StringTok{"Shank2"}\NormalTok{, }\StringTok{"Shank3"}\NormalTok{, }\StringTok{"Syngap1"}\NormalTok{, }\StringTok{"Ube3a"}\NormalTok{)}
\NormalTok{IRS_protein <-}\StringTok{ }\KeywordTok{normalize_IRS}\NormalTok{(SL_protein, }\StringTok{"QC"}\NormalTok{, groups, }\DataTypeTok{robust =} \OtherTok{TRUE}\NormalTok{)}
\end{Highlighting}
\end{Shaded}

\begin{verbatim}
## [1] "Used robust (geometric) mean."
\end{verbatim}

\subsection{Identify and remove QC
outliers.}\label{identify-and-remove-qc-outliers.}

IRS normalization utilizes QC samples as reference. Outlier QC
measurements (caused by interference or other artifact) would influence
the IRS normalizaiton. Thus, outlier QC samples are removed, if
identified.

\begin{Shaded}
\begin{Highlighting}[]
\CommentTok{# Data is...}
\NormalTok{data_in <-}\StringTok{ }\NormalTok{IRS_protein}

\CommentTok{# Illustrate Oldham's sample connectivity.}
\NormalTok{sample_connectivity <-}\StringTok{ }\KeywordTok{ggplotSampleConnectivityv2}\NormalTok{(IRS_protein, }\DataTypeTok{colID =} \StringTok{"QC"}\NormalTok{)}
\NormalTok{tab <-}\StringTok{ }\NormalTok{sample_connectivity}\OperatorTok{$}\NormalTok{table}
\NormalTok{df <-}\StringTok{ }\KeywordTok{add_column}\NormalTok{(tab,}\DataTypeTok{SampleName =} \KeywordTok{rownames}\NormalTok{(tab),}\DataTypeTok{.before =} \DecValTok{1}\NormalTok{)}
\KeywordTok{rownames}\NormalTok{(df) <-}\StringTok{ }\OtherTok{NULL}
\NormalTok{knitr}\OperatorTok{::}\KeywordTok{kable}\NormalTok{(df)}
\end{Highlighting}
\end{Shaded}

\begin{longtable}[]{@{}lrr@{}}
\toprule
SampleName & Z.Ki & Sample\tabularnewline
\midrule
\endhead
Abundance: F2: 131C, Sample, SPQC, 38901, Ube3a & -2.4285099 &
5\tabularnewline
Abundance: F2: 129N, Sample, SPQC, 38900, Ube3a & -0.6944381 &
3\tabularnewline
Abundance: F1: 129N, Sample, SPQC, 37847, Syngap1 & -0.6922050 &
4\tabularnewline
Abundance: F1: 126, Sample, SPQC, 37846, Syngap1 & -0.5299421 &
1\tabularnewline
Abundance: F2: 126, Sample, SPQC, 38899, Ube3a & 0.0355540 &
9\tabularnewline
Abundance: F4: 126, Sample, SPQC, 38910, Shank2 & 0.0427824 &
12\tabularnewline
Abundance: F3: 131C, Sample, SPQC, 41354, Shank3 & 0.1083114 &
7\tabularnewline
Abundance: F1: 131C, Sample, SPQC, 37848, Syngap1 & 0.4476606 &
10\tabularnewline
Abundance: F4: 129N, Sample, SPQC, 38911, Shank2 & 0.6434615 &
8\tabularnewline
Abundance: F3: 126, Sample, SPQC, 41352, Shank3 & 0.8178823 &
6\tabularnewline
Abundance: F3: 129N, Sample, SPQC, 41353, Shank3 & 0.9984023 &
11\tabularnewline
Abundance: F4: 131C, Sample, SPQC, 38912, Shank2 & 1.2510404 &
2\tabularnewline
\bottomrule
\end{longtable}

\begin{Shaded}
\begin{Highlighting}[]
\NormalTok{plot <-}\StringTok{ }\NormalTok{sample_connectivity}\OperatorTok{$}\NormalTok{connectivityplot }\OperatorTok{+}\StringTok{ }
\StringTok{  }\KeywordTok{ggtitle}\NormalTok{(}\StringTok{"QC Sample Connectivity"}\NormalTok{)}

\CommentTok{# Figure.}
\NormalTok{caption <-}\StringTok{ }\KeywordTok{strwrap}\NormalTok{(}\StringTok{"QC Sample Connectivity. Examination of QC samples Z-Score }
\StringTok{                  normalized connectivity as a means of identifying outlier }
\StringTok{                  QC samples."}\NormalTok{, }\DataTypeTok{width =} \OtherTok{Inf}\NormalTok{, }\DataTypeTok{simplify =} \OtherTok{TRUE}\NormalTok{)}
\end{Highlighting}
\end{Shaded}

\begin{figure}

{\centering \includegraphics{1_TMT_Analysis_files/figure-latex/unnamed-chunk-67-1} 

}

\caption{QC Sample Connectivity after outlier removal.}\label{fig:unnamed-chunk-67}
\end{figure}

\begin{verbatim}
## Saving 6.5 x 4.5 in image
\end{verbatim}

\begin{verbatim}
## [1] "0 outlier sample(s) identified in iteration 1."
## [1] "0 outlier sample(s) identified in iteration 2."
## [1] "0 outlier sample(s) identified in iteration 3."
## [1] "0 outlier sample(s) identified in iteration 4."
## [1] "0 outlier sample(s) identified in iteration 5."
\end{verbatim}

\begin{verbatim}
## [1] "none" "none" "none" "none" "none"
\end{verbatim}

\begin{verbatim}
## [1] "Used robust (geometric) mean."
\end{verbatim}

\begin{figure}

{\centering \includegraphics{1_TMT_Analysis_files/figure-latex/unnamed-chunk-68-1} 

}

\caption{QC Sample Connectivity after outlier removal.}\label{fig:unnamed-chunk-68}
\end{figure}

\subsection{Examine the IRS Normalized protein level
data.}\label{examine-the-irs-normalized-protein-level-data.}

\begin{Shaded}
\begin{Highlighting}[]
\NormalTok{data_in <-}\StringTok{ }\NormalTok{IRS_protein}
\NormalTok{title <-}\StringTok{ "IRS Normalized protein"}

\CommentTok{# Generate boxplot.}
\NormalTok{colors <-}\StringTok{ }\KeywordTok{c}\NormalTok{(}\KeywordTok{rep}\NormalTok{(}\StringTok{"green"}\NormalTok{, }\DecValTok{11}\NormalTok{), }\KeywordTok{rep}\NormalTok{(}\StringTok{"purple"}\NormalTok{, }\DecValTok{11}\NormalTok{), }
            \KeywordTok{rep}\NormalTok{(}\StringTok{"yellow"}\NormalTok{, }\DecValTok{11}\NormalTok{), }\KeywordTok{rep}\NormalTok{(}\StringTok{"blue"}\NormalTok{, }\DecValTok{11}\NormalTok{))}
\NormalTok{p1 <-}\StringTok{ }\KeywordTok{ggplotBoxPlot}\NormalTok{(data_in, }\DataTypeTok{colID =} \StringTok{"Abundance"}\NormalTok{, colors, title)}

\CommentTok{# Generate density plot.}
\NormalTok{p2 <-}\StringTok{ }\KeywordTok{ggplotDensity}\NormalTok{(data_in, }\DataTypeTok{colID =} \StringTok{"Abundance"}\NormalTok{, title) }\OperatorTok{+}\StringTok{ }
\StringTok{  }\KeywordTok{theme}\NormalTok{(}\DataTypeTok{legend.position =} \StringTok{"none"}\NormalTok{)}
\NormalTok{colors <-}\StringTok{ }\KeywordTok{c}\NormalTok{(}\KeywordTok{rep}\NormalTok{(}\StringTok{"yellow"}\NormalTok{, }\DecValTok{11}\NormalTok{), }\KeywordTok{rep}\NormalTok{(}\StringTok{"blue"}\NormalTok{, }\DecValTok{11}\NormalTok{), }
            \KeywordTok{rep}\NormalTok{(}\StringTok{"green"}\NormalTok{, }\DecValTok{11}\NormalTok{), }\KeywordTok{rep}\NormalTok{(}\StringTok{"purple"}\NormalTok{, }\DecValTok{11}\NormalTok{))}
\NormalTok{p2 <-}\StringTok{ }\NormalTok{p2 }\OperatorTok{+}\StringTok{ }\KeywordTok{scale_color_manual}\NormalTok{(}\DataTypeTok{values =}\NormalTok{ colors)}

\CommentTok{# Generate meanSd plot.}
\NormalTok{p3 <-}\StringTok{ }\KeywordTok{ggplotMeanSdPlot}\NormalTok{(data_in, }\DataTypeTok{colID =} \StringTok{"Abundance"}\NormalTok{, title, }\DataTypeTok{log =} \OtherTok{TRUE}\NormalTok{)}
\end{Highlighting}
\end{Shaded}

\begin{verbatim}
## Warning: Removed 1 rows containing non-finite values (stat_binhex).
\end{verbatim}

\begin{Shaded}
\begin{Highlighting}[]
\CommentTok{# Generate MDS plot.}
\NormalTok{colors <-}\StringTok{ }\KeywordTok{c}\NormalTok{(}\KeywordTok{rep}\NormalTok{(}\StringTok{"yellow"}\NormalTok{, }\DecValTok{3}\NormalTok{), }\KeywordTok{rep}\NormalTok{(}\StringTok{"blue"}\NormalTok{, }\DecValTok{3}\NormalTok{), }\KeywordTok{rep}\NormalTok{(}\StringTok{"green"}\NormalTok{, }\DecValTok{3}\NormalTok{), }\KeywordTok{rep}\NormalTok{(}\StringTok{"purple"}\NormalTok{, }\DecValTok{3}\NormalTok{))}
\NormalTok{p4 <-}\StringTok{ }\KeywordTok{ggplotMDS}\NormalTok{(data_in, }\DataTypeTok{colID =} \StringTok{"Abundance"}\NormalTok{, colors, title, sample_info, }\DataTypeTok{labels =} \OtherTok{TRUE}\NormalTok{) }\OperatorTok{+}
\StringTok{  }\KeywordTok{theme}\NormalTok{(}\DataTypeTok{legend.position =} \StringTok{"none"}\NormalTok{)}

\CommentTok{# Figure.}
\NormalTok{caption <-}\StringTok{ }\KeywordTok{strwrap}\NormalTok{(}\StringTok{"IRS normalized protein. A. Boxplot. B. Density plot.}
\StringTok{                   C. Mean SD plot. D. MDS plot"}\NormalTok{, }\DataTypeTok{width =} \OtherTok{Inf}\NormalTok{, }\DataTypeTok{simplify =} \OtherTok{TRUE}\NormalTok{)}
\end{Highlighting}
\end{Shaded}

\begin{figure}

{\centering \includegraphics{1_TMT_Analysis_files/figure-latex/unnamed-chunk-72-1} 

}

\caption{IRS normalized protein. A. Boxplot. B. Density plot.  C. Mean SD plot. D. MDS plot}\label{fig:unnamed-chunk-72}
\end{figure}

\begin{verbatim}
## Saving 6.5 x 4.5 in image
\end{verbatim}

\subsection{Protein level filtering, imputing, and final TMM
normalization.}\label{protein-level-filtering-imputing-and-final-tmm-normalization.}

\begin{Shaded}
\begin{Highlighting}[]
\CommentTok{# Remove proteins that are identified by only 1 peptide.}
\CommentTok{# Remove proteins identified in less than 50% of samples.}
\NormalTok{filt_protein <-}\StringTok{ }\KeywordTok{filter_proteins}\NormalTok{(IRS_protein, }\StringTok{"Abundance"}\NormalTok{)}
\end{Highlighting}
\end{Shaded}

\begin{verbatim}
## [1] "119 proteins are identified by only one peptide and will be removed."
## [1] "10 proteins are identified in less than 50% of samples and are removed."
\end{verbatim}

\begin{Shaded}
\begin{Highlighting}[]
\CommentTok{# Generate plot to examine distribution of remaining missing values.}
\NormalTok{plot <-}\StringTok{ }\KeywordTok{ggplotDetect}\NormalTok{(filt_protein, }\StringTok{"Abundance"}\NormalTok{) }\OperatorTok{+}
\StringTok{  }\KeywordTok{ggtitle}\NormalTok{(}\StringTok{"Protein missing value distribution"}\NormalTok{)}

\CommentTok{# Figure}
\NormalTok{caption <-}\StringTok{ }\KeywordTok{c}\NormalTok{(}\StringTok{"Protein level missing value distribution."}\NormalTok{)}
\end{Highlighting}
\end{Shaded}

\begin{verbatim}
## Cluster size 3363 broken into 2130 1233 
## Cluster size 2130 broken into 1407 723 
## Done cluster 1407 
## Done cluster 723 
## Done cluster 2130 
## Done cluster 1233
\end{verbatim}

\begin{verbatim}
## Saving 6.5 x 4.5 in image
\end{verbatim}

\subsection{Examine the TMM Normalized protein level
data.}\label{examine-the-tmm-normalized-protein-level-data.}

\begin{Shaded}
\begin{Highlighting}[]
\NormalTok{data_in <-}\StringTok{ }\NormalTok{TMM_protein}
\NormalTok{title <-}\StringTok{ "TMM Normalized protein"}

\CommentTok{# Generate boxplot.}
\CommentTok{# Adjust color vector if samples were removed.}
\CommentTok{# Cortex outliers = 1x Ube3a, and 1x Syngap1}
\NormalTok{colors <-}\StringTok{ }\KeywordTok{c}\NormalTok{(}\KeywordTok{rep}\NormalTok{(}\StringTok{"green"}\NormalTok{, }\DecValTok{11}\NormalTok{), }\KeywordTok{rep}\NormalTok{(}\StringTok{"purple"}\NormalTok{, }\DecValTok{11}\NormalTok{), }\KeywordTok{rep}\NormalTok{(}\StringTok{"yellow"}\NormalTok{, }\DecValTok{11}\NormalTok{), }\KeywordTok{rep}\NormalTok{(}\StringTok{"blue"}\NormalTok{, }\DecValTok{11}\NormalTok{))}
\NormalTok{p1 <-}\StringTok{ }\KeywordTok{ggplotBoxPlot}\NormalTok{(data_in, }\DataTypeTok{colID =} \StringTok{"Abundance"}\NormalTok{, colors, title)}

\CommentTok{# Generate density plot.}
\NormalTok{p2 <-}\StringTok{ }\KeywordTok{ggplotDensity}\NormalTok{(data_in, }\DataTypeTok{colID =} \StringTok{"Abundance"}\NormalTok{, title) }\OperatorTok{+}\StringTok{ }\KeywordTok{theme}\NormalTok{(}\DataTypeTok{legend.position =} \StringTok{"none"}\NormalTok{)}
\NormalTok{colors <-}\StringTok{ }\KeywordTok{c}\NormalTok{(}\KeywordTok{rep}\NormalTok{(}\StringTok{"yellow"}\NormalTok{, }\DecValTok{11}\NormalTok{), }\KeywordTok{rep}\NormalTok{(}\StringTok{"blue"}\NormalTok{, }\DecValTok{11}\NormalTok{), }\KeywordTok{rep}\NormalTok{(}\StringTok{"green"}\NormalTok{, }\DecValTok{11}\NormalTok{), }\KeywordTok{rep}\NormalTok{(}\StringTok{"purple"}\NormalTok{, }\DecValTok{11}\NormalTok{))}
\NormalTok{p2 <-}\StringTok{ }\NormalTok{p2 }\OperatorTok{+}\StringTok{ }\KeywordTok{scale_color_manual}\NormalTok{(}\DataTypeTok{values =}\NormalTok{ colors)}

\CommentTok{# Generate meanSd plot.}
\NormalTok{p3 <-}\StringTok{ }\KeywordTok{ggplotMeanSdPlot}\NormalTok{(data_in, }\DataTypeTok{colID =} \StringTok{"Abundance"}\NormalTok{, title, }\DataTypeTok{log =} \OtherTok{TRUE}\NormalTok{)}

\CommentTok{# Generate MDS plot.}
\NormalTok{colors <-}\StringTok{ }\KeywordTok{c}\NormalTok{(}\KeywordTok{rep}\NormalTok{(}\StringTok{"yellow"}\NormalTok{, }\DecValTok{3}\NormalTok{), }\KeywordTok{rep}\NormalTok{(}\StringTok{"blue"}\NormalTok{, }\DecValTok{3}\NormalTok{), }\KeywordTok{rep}\NormalTok{(}\StringTok{"green"}\NormalTok{, }\DecValTok{3}\NormalTok{), }\KeywordTok{rep}\NormalTok{(}\StringTok{"purple"}\NormalTok{, }\DecValTok{3}\NormalTok{))}
\NormalTok{p4 <-}\StringTok{ }\KeywordTok{ggplotMDS}\NormalTok{(data_in, }\DataTypeTok{colID =} \StringTok{"Abundance"}\NormalTok{, colors, title, sample_info, }\DataTypeTok{labels =} \OtherTok{TRUE}\NormalTok{) }\OperatorTok{+}
\StringTok{  }\KeywordTok{theme}\NormalTok{(}\DataTypeTok{legend.position =} \StringTok{"none"}\NormalTok{)}

\CommentTok{# Figure.}
\NormalTok{caption <-}\StringTok{ }\KeywordTok{strwrap}\NormalTok{(}\StringTok{"TMM normalized protein. A. Boxplot. B. Density plot.}
\StringTok{                   C. Mean SD plot. D. MDS plot"}\NormalTok{, }\DataTypeTok{width =} \OtherTok{Inf}\NormalTok{, }\DataTypeTok{simplify =} \OtherTok{TRUE}\NormalTok{)}
\end{Highlighting}
\end{Shaded}

\begin{figure}

{\centering \includegraphics{1_TMT_Analysis_files/figure-latex/unnamed-chunk-80-1} 

}

\caption{TMM normalized protein. A. Boxplot. B. Density plot.  C. Mean SD plot. D. MDS plot}\label{fig:unnamed-chunk-80}
\end{figure}

\begin{verbatim}
## Saving 6.5 x 4.5 in image
\end{verbatim}

\subsection{IntraBatch statistical testing: EdgeR
GLM.}\label{intrabatch-statistical-testing-edger-glm.}

\begin{Shaded}
\begin{Highlighting}[]
\CommentTok{# data is...}
\NormalTok{data_in <-}\StringTok{ }\NormalTok{TMM_protein }\CommentTok{# normalized, imputed, protein level data.}
\NormalTok{data_in[}\DecValTok{1}\OperatorTok{:}\DecValTok{5}\NormalTok{, }\DecValTok{1}\OperatorTok{:}\DecValTok{5}\NormalTok{]}
\end{Highlighting}
\end{Shaded}

\begin{verbatim}
##   Accession Peptides Abundance: F4: 126, Sample, SPQC, 38910, Shank2
## 1    A2A5R2       16                                       241145.69
## 2    A2A690       20                                       721806.74
## 3    A2A699       15                                       578377.03
## 4    A2A8L5       15                                       202235.86
## 5    A2AAE1        4                                        28034.55
##   Abundance: F4: 129N, Sample, SPQC, 38911, Shank2
## 1                                        249254.75
## 2                                        698435.19
## 3                                        597695.33
## 4                                        193536.45
## 5                                         30967.43
##   Abundance: F4: 131C, Sample, SPQC, 38912, Shank2
## 1                                        250369.86
## 2                                        711757.65
## 3                                        558300.22
## 4                                        212743.13
## 5                                         30340.11
\end{verbatim}

\begin{Shaded}
\begin{Highlighting}[]
\KeywordTok{dim}\NormalTok{(data_in)}
\end{Highlighting}
\end{Shaded}

\begin{verbatim}
## [1] 3363   46
\end{verbatim}

\begin{Shaded}
\begin{Highlighting}[]
\CommentTok{# Format data for EdgeR. Remove QC Samples!}
\NormalTok{cols <-}\StringTok{ }\KeywordTok{grepl}\NormalTok{(}\StringTok{"Abundance"}\NormalTok{, }\KeywordTok{colnames}\NormalTok{(TMM_protein))}
\NormalTok{dm <-}\StringTok{ }\KeywordTok{as.matrix}\NormalTok{(data_in[, cols])}
\KeywordTok{rownames}\NormalTok{(dm) <-}\StringTok{ }\NormalTok{data_in}\OperatorTok{$}\NormalTok{Accession}
\NormalTok{out <-}\StringTok{ }\KeywordTok{grepl}\NormalTok{(}\StringTok{"QC"}\NormalTok{, }\KeywordTok{colnames}\NormalTok{(dm))}
\NormalTok{dm <-}\StringTok{ }\NormalTok{dm[, }\OperatorTok{!}\NormalTok{out]}
\KeywordTok{dim}\NormalTok{(dm)}
\end{Highlighting}
\end{Shaded}

\begin{verbatim}
## [1] 3363   32
\end{verbatim}

\begin{Shaded}
\begin{Highlighting}[]
\CommentTok{# Check, there should be no missing values.}
\KeywordTok{sum}\NormalTok{(}\KeywordTok{is.na}\NormalTok{(dm)) }\OperatorTok{==}\StringTok{ }\DecValTok{0}
\end{Highlighting}
\end{Shaded}

\begin{verbatim}
## [1] TRUE
\end{verbatim}

\begin{Shaded}
\begin{Highlighting}[]
\CommentTok{# Check, traits and data in matching order?}
\NormalTok{traits <-}\StringTok{ }\NormalTok{sample_info}
\NormalTok{traits_sub <-}\StringTok{ }\KeywordTok{subset}\NormalTok{(traits, }\OperatorTok{!}\NormalTok{traits}\OperatorTok{$}\NormalTok{SampleType }\OperatorTok{==}\StringTok{ "QC"}\NormalTok{)}
\NormalTok{traits_sub <-}\StringTok{ }\KeywordTok{subset}\NormalTok{(traits_sub, traits_sub}\OperatorTok{$}\NormalTok{ColumnName }\OperatorTok\StringTok{ }\KeywordTok{colnames}\NormalTok{(dm))}
\KeywordTok{all}\NormalTok{(}\KeywordTok{colnames}\NormalTok{(dm) }\OperatorTok{==}\StringTok{ }\NormalTok{traits_sub}\OperatorTok{$}\NormalTok{ColumnName)}
\end{Highlighting}
\end{Shaded}

\begin{verbatim}
## [1] TRUE
\end{verbatim}

\begin{Shaded}
\begin{Highlighting}[]
\CommentTok{# Create DGEList object with mapping to genotype (group).}
\NormalTok{group <-}\StringTok{ }\KeywordTok{as.factor}\NormalTok{(traits_sub}\OperatorTok{$}\NormalTok{Sample.Model)}
\NormalTok{y_DGE <-}\StringTok{ }\KeywordTok{DGEList}\NormalTok{(}\DataTypeTok{counts =}\NormalTok{ dm, }\DataTypeTok{group =}\NormalTok{ group)}

\CommentTok{# Basic design matrix for GLM.}
\NormalTok{design <-}\StringTok{ }\KeywordTok{model.matrix}\NormalTok{(}\OperatorTok{~}\StringTok{ }\DecValTok{0} \OperatorTok{+}\StringTok{ }\NormalTok{group, }\DataTypeTok{data =}\NormalTok{ y_DGE}\OperatorTok{$}\NormalTok{samples)}
\KeywordTok{colnames}\NormalTok{(design)[}\DecValTok{1}\OperatorTok{:}\KeywordTok{length}\NormalTok{(}\KeywordTok{unique}\NormalTok{(traits_sub}\OperatorTok{$}\NormalTok{Sample.Model))] <-}\StringTok{ }\KeywordTok{levels}\NormalTok{(y_DGE}\OperatorTok{$}\NormalTok{samples}\OperatorTok{$}\NormalTok{group)}
\NormalTok{design}
\end{Highlighting}
\end{Shaded}

\begin{verbatim}
##                                                  HET.Syngap1 KO.Shank2
## Abundance: F4: 127N, Sample, WT, 38902, Shank2             0         0
## Abundance: F4: 128N, Sample, WT, 38903, Shank2             0         0
## Abundance: F4: 129C, Sample, WT, 38908, Shank2             0         0
## Abundance: F4: 130C, Sample, WT, 42373, Shank2             0         0
## Abundance: F4: 127C, Sample, KO, 38905, Shank2             0         1
## Abundance: F4: 128C, Sample, KO, 38907, Shank2             0         1
## Abundance: F4: 130N, Sample, KO, 38909, Shank2             0         1
## Abundance: F4: 131N, Sample, KO, 42372, Shank2             0         1
## Abundance: F3: 127N, Sample, WT, 41341, Shank3             0         0
## Abundance: F3: 128N, Sample, WT, 41342, Shank3             0         0
## Abundance: F3: 129C, Sample, WT, 41347, Shank3             0         0
## Abundance: F3: 130C, Sample, WT, 41348, Shank3             0         0
## Abundance: F3: 127C, Sample, KO, 41343, Shank3             0         0
## Abundance: F3: 128C, Sample, KO, 41344, Shank3             0         0
## Abundance: F3: 130N, Sample, KO, 41345, Shank3             0         0
## Abundance: F3: 131N, Sample, KO, 41346, Shank3             0         0
## Abundance: F1: 127N, Sample, WT, 37649, Syngap1            0         0
## Abundance: F1: 128N, Sample, WT, 37650, Syngap1            0         0
## Abundance: F1: 129C, Sample, WT, 37651, Syngap1            0         0
## Abundance: F1: 130C, Sample, WT, 37652, Syngap1            0         0
## Abundance: F1: 127C, Sample, HET, 37648, Syngap1           1         0
## Abundance: F1: 128C, Sample, HET, 37653, Syngap1           1         0
## Abundance: F1: 130N, Sample, HET, 37654, Syngap1           1         0
## Abundance: F1: 131N, Sample, HET, 37655, Syngap1           1         0
## Abundance: F2: 127N, Sample, WT, 38891, Ube3a              0         0
## Abundance: F2: 128N, Sample, WT, 38892, Ube3a              0         0
## Abundance: F2: 129C, Sample, WT, 38895, Ube3a              0         0
## Abundance: F2: 130C, Sample, WT, 38896, Ube3a              0         0
## Abundance: F2: 127C, Sample, KO, 38893, Ube3a              0         0
## Abundance: F2: 128C, Sample, KO, 38894, Ube3a              0         0
## Abundance: F2: 130N, Sample, KO, 38897, Ube3a              0         0
## Abundance: F2: 131N, Sample, KO, 38898, Ube3a              0         0
##                                                  KO.Shank3 KO.Ube3a
## Abundance: F4: 127N, Sample, WT, 38902, Shank2           0        0
## Abundance: F4: 128N, Sample, WT, 38903, Shank2           0        0
## Abundance: F4: 129C, Sample, WT, 38908, Shank2           0        0
## Abundance: F4: 130C, Sample, WT, 42373, Shank2           0        0
## Abundance: F4: 127C, Sample, KO, 38905, Shank2           0        0
## Abundance: F4: 128C, Sample, KO, 38907, Shank2           0        0
## Abundance: F4: 130N, Sample, KO, 38909, Shank2           0        0
## Abundance: F4: 131N, Sample, KO, 42372, Shank2           0        0
## Abundance: F3: 127N, Sample, WT, 41341, Shank3           0        0
## Abundance: F3: 128N, Sample, WT, 41342, Shank3           0        0
## Abundance: F3: 129C, Sample, WT, 41347, Shank3           0        0
## Abundance: F3: 130C, Sample, WT, 41348, Shank3           0        0
## Abundance: F3: 127C, Sample, KO, 41343, Shank3           1        0
## Abundance: F3: 128C, Sample, KO, 41344, Shank3           1        0
## Abundance: F3: 130N, Sample, KO, 41345, Shank3           1        0
## Abundance: F3: 131N, Sample, KO, 41346, Shank3           1        0
## Abundance: F1: 127N, Sample, WT, 37649, Syngap1          0        0
## Abundance: F1: 128N, Sample, WT, 37650, Syngap1          0        0
## Abundance: F1: 129C, Sample, WT, 37651, Syngap1          0        0
## Abundance: F1: 130C, Sample, WT, 37652, Syngap1          0        0
## Abundance: F1: 127C, Sample, HET, 37648, Syngap1         0        0
## Abundance: F1: 128C, Sample, HET, 37653, Syngap1         0        0
## Abundance: F1: 130N, Sample, HET, 37654, Syngap1         0        0
## Abundance: F1: 131N, Sample, HET, 37655, Syngap1         0        0
## Abundance: F2: 127N, Sample, WT, 38891, Ube3a            0        0
## Abundance: F2: 128N, Sample, WT, 38892, Ube3a            0        0
## Abundance: F2: 129C, Sample, WT, 38895, Ube3a            0        0
## Abundance: F2: 130C, Sample, WT, 38896, Ube3a            0        0
## Abundance: F2: 127C, Sample, KO, 38893, Ube3a            0        1
## Abundance: F2: 128C, Sample, KO, 38894, Ube3a            0        1
## Abundance: F2: 130N, Sample, KO, 38897, Ube3a            0        1
## Abundance: F2: 131N, Sample, KO, 38898, Ube3a            0        1
##                                                  WT.Shank2 WT.Shank3
## Abundance: F4: 127N, Sample, WT, 38902, Shank2           1         0
## Abundance: F4: 128N, Sample, WT, 38903, Shank2           1         0
## Abundance: F4: 129C, Sample, WT, 38908, Shank2           1         0
## Abundance: F4: 130C, Sample, WT, 42373, Shank2           1         0
## Abundance: F4: 127C, Sample, KO, 38905, Shank2           0         0
## Abundance: F4: 128C, Sample, KO, 38907, Shank2           0         0
## Abundance: F4: 130N, Sample, KO, 38909, Shank2           0         0
## Abundance: F4: 131N, Sample, KO, 42372, Shank2           0         0
## Abundance: F3: 127N, Sample, WT, 41341, Shank3           0         1
## Abundance: F3: 128N, Sample, WT, 41342, Shank3           0         1
## Abundance: F3: 129C, Sample, WT, 41347, Shank3           0         1
## Abundance: F3: 130C, Sample, WT, 41348, Shank3           0         1
## Abundance: F3: 127C, Sample, KO, 41343, Shank3           0         0
## Abundance: F3: 128C, Sample, KO, 41344, Shank3           0         0
## Abundance: F3: 130N, Sample, KO, 41345, Shank3           0         0
## Abundance: F3: 131N, Sample, KO, 41346, Shank3           0         0
## Abundance: F1: 127N, Sample, WT, 37649, Syngap1          0         0
## Abundance: F1: 128N, Sample, WT, 37650, Syngap1          0         0
## Abundance: F1: 129C, Sample, WT, 37651, Syngap1          0         0
## Abundance: F1: 130C, Sample, WT, 37652, Syngap1          0         0
## Abundance: F1: 127C, Sample, HET, 37648, Syngap1         0         0
## Abundance: F1: 128C, Sample, HET, 37653, Syngap1         0         0
## Abundance: F1: 130N, Sample, HET, 37654, Syngap1         0         0
## Abundance: F1: 131N, Sample, HET, 37655, Syngap1         0         0
## Abundance: F2: 127N, Sample, WT, 38891, Ube3a            0         0
## Abundance: F2: 128N, Sample, WT, 38892, Ube3a            0         0
## Abundance: F2: 129C, Sample, WT, 38895, Ube3a            0         0
## Abundance: F2: 130C, Sample, WT, 38896, Ube3a            0         0
## Abundance: F2: 127C, Sample, KO, 38893, Ube3a            0         0
## Abundance: F2: 128C, Sample, KO, 38894, Ube3a            0         0
## Abundance: F2: 130N, Sample, KO, 38897, Ube3a            0         0
## Abundance: F2: 131N, Sample, KO, 38898, Ube3a            0         0
##                                                  WT.Syngap1 WT.Ube3a
## Abundance: F4: 127N, Sample, WT, 38902, Shank2            0        0
## Abundance: F4: 128N, Sample, WT, 38903, Shank2            0        0
## Abundance: F4: 129C, Sample, WT, 38908, Shank2            0        0
## Abundance: F4: 130C, Sample, WT, 42373, Shank2            0        0
## Abundance: F4: 127C, Sample, KO, 38905, Shank2            0        0
## Abundance: F4: 128C, Sample, KO, 38907, Shank2            0        0
## Abundance: F4: 130N, Sample, KO, 38909, Shank2            0        0
## Abundance: F4: 131N, Sample, KO, 42372, Shank2            0        0
## Abundance: F3: 127N, Sample, WT, 41341, Shank3            0        0
## Abundance: F3: 128N, Sample, WT, 41342, Shank3            0        0
## Abundance: F3: 129C, Sample, WT, 41347, Shank3            0        0
## Abundance: F3: 130C, Sample, WT, 41348, Shank3            0        0
## Abundance: F3: 127C, Sample, KO, 41343, Shank3            0        0
## Abundance: F3: 128C, Sample, KO, 41344, Shank3            0        0
## Abundance: F3: 130N, Sample, KO, 41345, Shank3            0        0
## Abundance: F3: 131N, Sample, KO, 41346, Shank3            0        0
## Abundance: F1: 127N, Sample, WT, 37649, Syngap1           1        0
## Abundance: F1: 128N, Sample, WT, 37650, Syngap1           1        0
## Abundance: F1: 129C, Sample, WT, 37651, Syngap1           1        0
## Abundance: F1: 130C, Sample, WT, 37652, Syngap1           1        0
## Abundance: F1: 127C, Sample, HET, 37648, Syngap1          0        0
## Abundance: F1: 128C, Sample, HET, 37653, Syngap1          0        0
## Abundance: F1: 130N, Sample, HET, 37654, Syngap1          0        0
## Abundance: F1: 131N, Sample, HET, 37655, Syngap1          0        0
## Abundance: F2: 127N, Sample, WT, 38891, Ube3a             0        1
## Abundance: F2: 128N, Sample, WT, 38892, Ube3a             0        1
## Abundance: F2: 129C, Sample, WT, 38895, Ube3a             0        1
## Abundance: F2: 130C, Sample, WT, 38896, Ube3a             0        1
## Abundance: F2: 127C, Sample, KO, 38893, Ube3a             0        0
## Abundance: F2: 128C, Sample, KO, 38894, Ube3a             0        0
## Abundance: F2: 130N, Sample, KO, 38897, Ube3a             0        0
## Abundance: F2: 131N, Sample, KO, 38898, Ube3a             0        0
## attr(,"assign")
## [1] 1 1 1 1 1 1 1 1
## attr(,"contrasts")
## attr(,"contrasts")$group
## [1] "contr.treatment"
\end{verbatim}

\begin{Shaded}
\begin{Highlighting}[]
\CommentTok{# Estimate dispersion:}
\NormalTok{y_DGE <-}\StringTok{ }\KeywordTok{estimateDisp}\NormalTok{(y_DGE, design, }\DataTypeTok{robust =} \OtherTok{TRUE}\NormalTok{)}

\CommentTok{# PlotBCV}
\NormalTok{plot <-}\StringTok{ }\KeywordTok{ggplotBCV}\NormalTok{(y_DGE)}
\NormalTok{plot}
\end{Highlighting}
\end{Shaded}

\includegraphics{1_TMT_Analysis_files/figure-latex/unnamed-chunk-83-1.pdf}

\begin{Shaded}
\begin{Highlighting}[]
\CommentTok{# Fit a general linear model.}
\NormalTok{fit <-}\StringTok{ }\KeywordTok{glmQLFit}\NormalTok{(y_DGE, design, }\DataTypeTok{robust =} \OtherTok{TRUE}\NormalTok{)}

\CommentTok{# Examine the QL fitted dispersion.}
\KeywordTok{plotQLDisp}\NormalTok{(fit)}
\end{Highlighting}
\end{Shaded}

\includegraphics{1_TMT_Analysis_files/figure-latex/unnamed-chunk-83-2.pdf}

\begin{Shaded}
\begin{Highlighting}[]
\CommentTok{# Create a list of contrasts for pairwise comparisons.}
\NormalTok{contrasts <-}\StringTok{ }\KeywordTok{list}\NormalTok{(}
\NormalTok{  WTvKO.Shank2 <-}\StringTok{ }\KeywordTok{makeContrasts}\NormalTok{(KO.Shank2 }\OperatorTok{-}\StringTok{ }\NormalTok{WT.Shank2, }\DataTypeTok{levels =}\NormalTok{ design),}
\NormalTok{  WTvKO.Shank3 <-}\StringTok{ }\KeywordTok{makeContrasts}\NormalTok{(KO.Shank3 }\OperatorTok{-}\StringTok{ }\NormalTok{WT.Shank3, }\DataTypeTok{levels =}\NormalTok{ design),}
\NormalTok{  WTvHET.Syngap1 <-}\StringTok{ }\KeywordTok{makeContrasts}\NormalTok{(HET.Syngap1 }\OperatorTok{-}\StringTok{ }\NormalTok{WT.Syngap1, }\DataTypeTok{levels =}\NormalTok{ design),}
\NormalTok{  WTvKO.Ube3a <-}\StringTok{ }\KeywordTok{makeContrasts}\NormalTok{(KO.Ube3a }\OperatorTok{-}\StringTok{ }\NormalTok{WT.Ube3a, }\DataTypeTok{levels =}\NormalTok{ design)}
\NormalTok{)}

\CommentTok{# Call glmQLFTest() to evaluate differences in contrasts.}
\NormalTok{qlf <-}\StringTok{ }\KeywordTok{lapply}\NormalTok{(contrasts, }\ControlFlowTok{function}\NormalTok{(x) }\KeywordTok{glmQLFTest}\NormalTok{(fit, }\DataTypeTok{contrast =}\NormalTok{ x))}

\NormalTok{## Determine number of significant results with decideTests().}
\CommentTok{#  Default FDR is 0.05.}
\NormalTok{summary_table <-}\StringTok{ }\KeywordTok{lapply}\NormalTok{(qlf, }\ControlFlowTok{function}\NormalTok{(x) }\KeywordTok{summary}\NormalTok{(}\KeywordTok{decideTests}\NormalTok{(x)))}
\NormalTok{overall <-}\StringTok{ }\KeywordTok{t}\NormalTok{(}\KeywordTok{matrix}\NormalTok{(}\KeywordTok{unlist}\NormalTok{(summary_table), }\DataTypeTok{nrow =} \DecValTok{3}\NormalTok{, }\DataTypeTok{ncol =} \DecValTok{4}\NormalTok{))}
\KeywordTok{rownames}\NormalTok{(overall) <-}\StringTok{ }\KeywordTok{unlist}\NormalTok{(}\KeywordTok{lapply}\NormalTok{(contrasts, }\ControlFlowTok{function}\NormalTok{(x) }\KeywordTok{colnames}\NormalTok{(x)))}
\KeywordTok{colnames}\NormalTok{(overall) <-}\StringTok{ }\KeywordTok{c}\NormalTok{(}\StringTok{"Down"}\NormalTok{, }\StringTok{"NotSig"}\NormalTok{, }\StringTok{"Up"}\NormalTok{)}
\NormalTok{overall <-}\StringTok{ }\KeywordTok{as.data.frame}\NormalTok{(overall)}
\NormalTok{overall <-}\StringTok{ }\KeywordTok{add_column}\NormalTok{(overall, }\DataTypeTok{Contrast =} \KeywordTok{rownames}\NormalTok{(overall), }\DataTypeTok{.before =} \DecValTok{1}\NormalTok{)}
\NormalTok{overall <-}\StringTok{ }\NormalTok{overall[, }\KeywordTok{c}\NormalTok{(}\DecValTok{1}\NormalTok{, }\DecValTok{3}\NormalTok{, }\DecValTok{2}\NormalTok{, }\DecValTok{4}\NormalTok{)]}
\NormalTok{overall}\OperatorTok{$}\NormalTok{TotalSig <-}\StringTok{ }\KeywordTok{rowSums}\NormalTok{(overall[, }\KeywordTok{c}\NormalTok{(}\DecValTok{3}\NormalTok{, }\DecValTok{4}\NormalTok{)])}
\CommentTok{# Table of DE candidates.}
\NormalTok{table <-}\StringTok{ }\KeywordTok{tableGrob}\NormalTok{(overall, }\DataTypeTok{rows =} \OtherTok{NULL}\NormalTok{)}
\KeywordTok{grid.arrange}\NormalTok{(table)}
\end{Highlighting}
\end{Shaded}

\includegraphics{1_TMT_Analysis_files/figure-latex/unnamed-chunk-83-3.pdf}

\begin{Shaded}
\begin{Highlighting}[]
\CommentTok{# Save table.}
\NormalTok{file <-}\StringTok{ }\KeywordTok{paste0}\NormalTok{(outputfigsdir, }\StringTok{"/"}\NormalTok{, outputMatName, }\StringTok{"_InraBatch_GLM_Table.tiff"}\NormalTok{)}
\KeywordTok{ggsave}\NormalTok{(file,table)}
\end{Highlighting}
\end{Shaded}

\begin{verbatim}
## Saving 6.5 x 4.5 in image
\end{verbatim}

\begin{Shaded}
\begin{Highlighting}[]
\CommentTok{# Call topTags to add FDR. Gather tablurized results.}
\NormalTok{results <-}\StringTok{ }\KeywordTok{lapply}\NormalTok{(qlf, }\ControlFlowTok{function}\NormalTok{(x) }\KeywordTok{topTags}\NormalTok{(x, }\DataTypeTok{n =} \OtherTok{Inf}\NormalTok{, }\DataTypeTok{sort.by =} \StringTok{"none"}\NormalTok{)}\OperatorTok{$}\NormalTok{table)}

\CommentTok{# Convert logCPM column to percent WT.}
\CommentTok{# Annotate with candidate column.}
\NormalTok{results <-}\StringTok{ }\KeywordTok{lapply}\NormalTok{(results, }\ControlFlowTok{function}\NormalTok{(x) }\KeywordTok{annotateTopTags}\NormalTok{(x))}

\CommentTok{# Annotate with Gene names and Entrez IDS.}
\NormalTok{results <-}\StringTok{ }\KeywordTok{lapply}\NormalTok{(results, }\ControlFlowTok{function}\NormalTok{(x) }\KeywordTok{annotate_Entrez}\NormalTok{(x))}
\end{Highlighting}
\end{Shaded}

\begin{verbatim}
## 'select()' returned 1:many mapping between keys and columns
\end{verbatim}

\begin{verbatim}
## 'select()' returned 1:many mapping between keys and columns
## 'select()' returned 1:many mapping between keys and columns
## 'select()' returned 1:many mapping between keys and columns
## 'select()' returned 1:many mapping between keys and columns
## 'select()' returned 1:many mapping between keys and columns
## 'select()' returned 1:many mapping between keys and columns
## 'select()' returned 1:many mapping between keys and columns
\end{verbatim}

\begin{Shaded}
\begin{Highlighting}[]
\CommentTok{# Add fitted data to results. Sort by p-value.}
\KeywordTok{names}\NormalTok{(results) <-}\StringTok{ }\NormalTok{groups}
\ControlFlowTok{for}\NormalTok{ (i }\ControlFlowTok{in} \DecValTok{1}\OperatorTok{:}\KeywordTok{length}\NormalTok{(groups)) \{}
\NormalTok{  cols <-}\StringTok{ }\KeywordTok{grepl}\NormalTok{(groups[i], }\KeywordTok{colnames}\NormalTok{(fit}\OperatorTok{$}\NormalTok{fitted.values))}
\NormalTok{  results[[i]] <-}\StringTok{ }\KeywordTok{merge}\NormalTok{(results[[i]],}
    \KeywordTok{log2}\NormalTok{(fit}\OperatorTok{$}\NormalTok{fitted.values[, cols]),}
    \DataTypeTok{by =} \StringTok{"row.names"}
\NormalTok{  )}
\NormalTok{  results[[i]]}\OperatorTok{$}\NormalTok{Row.names <-}\StringTok{ }\OtherTok{NULL}
\NormalTok{  results[[i]] <-}\StringTok{ }\NormalTok{results[[i]][}\KeywordTok{order}\NormalTok{(results[[i]]}\OperatorTok{$}\NormalTok{FDR), ]}
\NormalTok{\}}

\CommentTok{# Pvalue histograms.}
\NormalTok{p1 <-}\StringTok{ }\KeywordTok{ggplotPvalHist}\NormalTok{(results[[}\DecValTok{1}\NormalTok{]], }\StringTok{"gold1"}\NormalTok{, }\StringTok{"Shank2"}\NormalTok{)}
\NormalTok{p2 <-}\StringTok{ }\KeywordTok{ggplotPvalHist}\NormalTok{(results[[}\DecValTok{2}\NormalTok{]], }\StringTok{"blue"}\NormalTok{, }\StringTok{"Shank3"}\NormalTok{)}
\NormalTok{p3 <-}\StringTok{ }\KeywordTok{ggplotPvalHist}\NormalTok{(results[[}\DecValTok{3}\NormalTok{]], }\StringTok{"green"}\NormalTok{, }\StringTok{"Syngap1"}\NormalTok{)}
\NormalTok{p4 <-}\StringTok{ }\KeywordTok{ggplotPvalHist}\NormalTok{(results[[}\DecValTok{4}\NormalTok{]], }\StringTok{"purple"}\NormalTok{, }\StringTok{"Ube3a"}\NormalTok{)}
\NormalTok{fig <-}\StringTok{ }\KeywordTok{plot_grid}\NormalTok{(p1, p2, p3, p4, }\DataTypeTok{labels =} \StringTok{"auto"}\NormalTok{)}
\NormalTok{fig}
\end{Highlighting}
\end{Shaded}

\includegraphics{1_TMT_Analysis_files/figure-latex/unnamed-chunk-83-4.pdf}

\begin{Shaded}
\begin{Highlighting}[]
\CommentTok{# Save as tiff.}
\NormalTok{file <-}\StringTok{ }\KeywordTok{paste0}\NormalTok{(outputfigsdir, }\StringTok{"/"}\NormalTok{, outputMatName, }\StringTok{"_IntraBatch_GLM_PvalueHist.tiff"}\NormalTok{)}
\KeywordTok{ggsave}\NormalTok{(file,fig)}
\end{Highlighting}
\end{Shaded}

\begin{verbatim}
## Saving 6.5 x 4.5 in image
\end{verbatim}

\begin{Shaded}
\begin{Highlighting}[]
\CommentTok{# Add summary table to results.}
\NormalTok{overall <-}\StringTok{ }\KeywordTok{list}\NormalTok{(overall)}
\KeywordTok{names}\NormalTok{(overall) <-}\StringTok{ "Summary"}
\NormalTok{results <-}\StringTok{ }\KeywordTok{c}\NormalTok{(overall, results)}
\NormalTok{results_intraBatch <-}\StringTok{ }\NormalTok{results}

\CommentTok{# Save workbook.}
\NormalTok{file <-}\StringTok{ }\KeywordTok{paste0}\NormalTok{(outputtabsdir, }\StringTok{"/"}\NormalTok{, outputMatName, }\StringTok{"_IntraBatch_GLM_results.xlsx"}\NormalTok{)}
\KeywordTok{write.excel}\NormalTok{(results, file)}

\CommentTok{# Save figures.}
\CommentTok{#file <- paste0(outputfigsdir, "/", outputMatName, "_IntraBatch_GLM_DE_Summary.pdf")}
\CommentTok{#ggsavePDF(plots = list(table, p1, p2, p3, p4), file)}

\CommentTok{# Save to RDS.}
\NormalTok{file <-}\StringTok{ }\KeywordTok{paste0}\NormalTok{(Rdatadir, }\StringTok{"/"}\NormalTok{, outputMatName, }\StringTok{"_IntraBatch_Results.RDS"}\NormalTok{)}
\KeywordTok{saveRDS}\NormalTok{(results_intraBatch, file)}
\end{Highlighting}
\end{Shaded}

\subsection{IntraBatch GO and KEGG enrichment testing with
EdgeR.}\label{intrabatch-go-and-kegg-enrichment-testing-with-edger.}

GO and KEGG enrichment analyis are performed using EdgeR's goana() and
kegga() functions.

\begin{Shaded}
\begin{Highlighting}[]
\NormalTok{## Perform GO and KEGG testing.}
\CommentTok{# edgeR_GSE() is a wrapper around the goana() and kegga() functions from EdgeR.}
\CommentTok{# This function operates on the QLF or ET objects.Rownames should be UniprotIDs.}
\CommentTok{# Proteins with an FDR <0.1 are be considered significant.}

\CommentTok{# Use lapply to generate GSE results.}
\CommentTok{# This will take a few minutes.}
\NormalTok{GSE_results <-}\StringTok{ }\KeywordTok{lapply}\NormalTok{(qlf, }\ControlFlowTok{function}\NormalTok{(x) }\KeywordTok{edgeR_GSE}\NormalTok{(x, }\DataTypeTok{FDR =} \FloatTok{0.1}\NormalTok{, }\DataTypeTok{filter =} \OtherTok{TRUE}\NormalTok{))}

\CommentTok{# Name GSE_results.}
\NormalTok{GSE_results <-}\StringTok{ }\KeywordTok{unlist}\NormalTok{(GSE_results, }\DataTypeTok{recursive =} \OtherTok{FALSE}\NormalTok{)}
\KeywordTok{names}\NormalTok{(GSE_results) <-}\StringTok{ }\KeywordTok{paste}\NormalTok{(}\KeywordTok{rep}\NormalTok{(groups, }\DataTypeTok{each =} \DecValTok{2}\NormalTok{), }\KeywordTok{names}\NormalTok{(GSE_results))}

\CommentTok{# Save workbook.}
\NormalTok{file <-}\StringTok{ }\KeywordTok{paste0}\NormalTok{(outputtabsdir, }\StringTok{"/"}\NormalTok{, outputMatName, }\StringTok{"_IntraBatch_GLM_GSE_Results.xlsx"}\NormalTok{)}
\KeywordTok{write.excel}\NormalTok{(GSE_results, file)}
\end{Highlighting}
\end{Shaded}

\subsection{Perform moderated EB regression of genetic strain as a
covariate.}\label{perform-moderated-eb-regression-of-genetic-strain-as-a-covariate.}

Moderated Empirical Bayes (EB) regression is performed to remove the
affect of genetic background.

\begin{Shaded}
\begin{Highlighting}[]
\CommentTok{# Data is...}
\CommentTok{# Check, there should be no missing values.}
\KeywordTok{sum}\NormalTok{(}\KeywordTok{is.na}\NormalTok{(TMM_protein)) }\OperatorTok{==}\StringTok{ }\DecValTok{0}
\NormalTok{data_in <-}\StringTok{ }\KeywordTok{na.omit}\NormalTok{(TMM_protein)}
\KeywordTok{dim}\NormalTok{(data_in)}
\NormalTok{colID <-}\StringTok{ "Abundance"}
\NormalTok{traits <-}\StringTok{ }\NormalTok{sample_info}

\CommentTok{# Remove the QC Data}
\NormalTok{out <-}\StringTok{ }\KeywordTok{grepl}\NormalTok{(}\StringTok{"QC"}\NormalTok{, }\KeywordTok{colnames}\NormalTok{(data_in))}
\NormalTok{data <-}\StringTok{ }\NormalTok{data_in[, }\OperatorTok{!}\NormalTok{out]}
\NormalTok{traits <-}\StringTok{ }\KeywordTok{subset}\NormalTok{(traits, }\OperatorTok{!}\NormalTok{traits}\OperatorTok{$}\NormalTok{SampleType }\OperatorTok{==}\StringTok{ "QC"}\NormalTok{)}
\NormalTok{traits <-}\StringTok{ }\KeywordTok{subset}\NormalTok{(traits, traits}\OperatorTok{$}\NormalTok{ColumnName }\OperatorTok\StringTok{ }\KeywordTok{colnames}\NormalTok{(data_in))}

\CommentTok{# Format data for EBLM}
\NormalTok{cols <-}\StringTok{ }\KeywordTok{grep}\NormalTok{(}\StringTok{"Abundance"}\NormalTok{, }\KeywordTok{colnames}\NormalTok{(data))}
\NormalTok{data <-}\StringTok{ }\KeywordTok{log2}\NormalTok{(}\KeywordTok{as.matrix}\NormalTok{(data[, cols]))}
\KeywordTok{rownames}\NormalTok{(data) <-}\StringTok{ }\NormalTok{data_in}\OperatorTok{$}\NormalTok{Accession}
\NormalTok{data <-}\StringTok{ }\KeywordTok{t}\NormalTok{(data)}
\NormalTok{data[}\DecValTok{1}\OperatorTok{:}\DecValTok{5}\NormalTok{, }\DecValTok{1}\OperatorTok{:}\DecValTok{5}\NormalTok{]}
\KeywordTok{dim}\NormalTok{(data)}

\CommentTok{# Check, data and traits are in matching order?}
\KeywordTok{all}\NormalTok{(}\KeywordTok{rownames}\NormalTok{(data) }\OperatorTok{==}\StringTok{ }\NormalTok{traits}\OperatorTok{$}\NormalTok{ColumnName)}

\CommentTok{# Define covariates.}
\NormalTok{status <-}\StringTok{ }\NormalTok{traits}\OperatorTok{$}\NormalTok{SampleType}
\NormalTok{sex <-}\StringTok{ }\KeywordTok{as.factor}\NormalTok{(traits}\OperatorTok{$}\NormalTok{Sex)}
\NormalTok{age <-}\StringTok{ }\KeywordTok{as.numeric}\NormalTok{(traits}\OperatorTok{$}\NormalTok{Age)}
\NormalTok{batch <-}\StringTok{ }\KeywordTok{as.factor}\NormalTok{(traits}\OperatorTok{$}\NormalTok{PrepDate)}
\NormalTok{strain <-}\StringTok{ }\KeywordTok{as.factor}\NormalTok{(traits}\OperatorTok{$}\NormalTok{Model)}

\CommentTok{# Design, we will perform regression on strain (mouse genetic background).}
\NormalTok{design <-}\StringTok{ }\KeywordTok{as.data.frame}\NormalTok{(}\KeywordTok{cbind}\NormalTok{(status, strain, batch, sex, age))}
\NormalTok{covariates <-}\StringTok{ }\KeywordTok{cbind}\NormalTok{(design}\OperatorTok{$}\NormalTok{strain)}

\CommentTok{# Eblm regression.}
\NormalTok{fit.eblm <-}\StringTok{ }\KeywordTok{empiricalBayesLM}\NormalTok{(data,}
  \DataTypeTok{removedCovariates =}\NormalTok{ covariates,}
  \DataTypeTok{fitToSamples =}\NormalTok{ design}\OperatorTok{$}\NormalTok{status }\OperatorTok{==}\StringTok{ "WT"}
\NormalTok{)}

\CommentTok{# Get fitted data.}
\NormalTok{data.fit <-}\StringTok{ }\NormalTok{fit.eblm}\OperatorTok{$}\NormalTok{adjustedData}

\CommentTok{# Examine overall PCA before and after EB regression.}
\NormalTok{colors <-}\StringTok{ }\NormalTok{traits}\OperatorTok{$}\NormalTok{Color}
\NormalTok{plot1 <-}\StringTok{ }\KeywordTok{ggplotPCA}\NormalTok{(}\KeywordTok{t}\NormalTok{(data), traits, colors, }\DataTypeTok{title =} \StringTok{"2D PCA Plot (Pre-Regression)"}\NormalTok{)}
\NormalTok{plot2 <-}\StringTok{ }\KeywordTok{ggplotPCA}\NormalTok{(}\KeywordTok{t}\NormalTok{(data.fit), traits, colors, }\DataTypeTok{title =} \StringTok{"2D PCA Plot (Post-Regression)"}\NormalTok{)}
\NormalTok{fig <-}\StringTok{ }\KeywordTok{plot_grid}\NormalTok{(plot1, plot2)}
\NormalTok{fig }

\CommentTok{# Just post-regression:}
\NormalTok{plot2}

\CommentTok{# Save plots.}
\CommentTok{#file <- paste0(outputfigsdir, "/", outputMatName, "_InterBatch_eBLM_Regression_PCA.pdf")}
\CommentTok{#ggsavePDF(plots = list(plot1, plot2), file)}

\CommentTok{# Save tiffs. }
\NormalTok{files <-}\StringTok{ }\KeywordTok{paste0}\NormalTok{(outputfigsdir, }\StringTok{"/"}\NormalTok{, outputMatName, }\KeywordTok{c}\NormalTok{(}\StringTok{"pre"}\NormalTok{,}\StringTok{"post"}\NormalTok{),}\StringTok{"_InterBatch_eBLM_Regression_PCA.tiff"}\NormalTok{)}
\KeywordTok{quiet}\NormalTok{(}\KeywordTok{mapply}\NormalTok{(ggsave,files,}\KeywordTok{list}\NormalTok{(plot1,plot2)))}
\end{Highlighting}
\end{Shaded}

\subsection{Reformat final normalized, regressed data for TAMPOR
processing.}\label{reformat-final-normalized-regressed-data-for-tampor-processing.}

\begin{Shaded}
\begin{Highlighting}[]
\CommentTok{# Data is...}
\NormalTok{data_in <-}\StringTok{ }\DecValTok{2}\OperatorTok{^}\KeywordTok{t}\NormalTok{(data.fit)}
\NormalTok{data_in[}\DecValTok{1}\OperatorTok{:}\DecValTok{5}\NormalTok{, }\DecValTok{1}\OperatorTok{:}\DecValTok{5}\NormalTok{]  }\CommentTok{# un-log}
\KeywordTok{dim}\NormalTok{(data_in)}

\CommentTok{# Extract Gene descriptions and uniprot accesssions for renaming rows.}
\NormalTok{idx <-}\StringTok{ }\KeywordTok{match}\NormalTok{(}\KeywordTok{rownames}\NormalTok{(data_in), SL_peptide}\OperatorTok{$}\NormalTok{Accession)}
\NormalTok{descriptions <-}\StringTok{ }\NormalTok{SL_peptide}\OperatorTok{$}\NormalTok{Description[idx]}
\NormalTok{uniprot <-}\StringTok{ }\NormalTok{SL_peptide}\OperatorTok{$}\NormalTok{Accession[idx]}

\CommentTok{# Split at "GN=", extract the second element.}
\NormalTok{long_names <-}\StringTok{ }\KeywordTok{sapply}\NormalTok{(}\KeywordTok{strsplit}\NormalTok{(descriptions, }\StringTok{"GN="}\NormalTok{, }\DataTypeTok{fixed =} \OtherTok{TRUE}\NormalTok{), }\StringTok{"["}\NormalTok{, }\DecValTok{2}\NormalTok{)}

\CommentTok{# Split at " ", extract the first element.}
\NormalTok{gene_names <-}\StringTok{ }\KeywordTok{sapply}\NormalTok{(}\KeywordTok{strsplit}\NormalTok{(long_names, }\StringTok{"}\CharTok{\textbackslash{}\textbackslash{}}\StringTok{ "}\NormalTok{), }\StringTok{"["}\NormalTok{, }\DecValTok{1}\NormalTok{)}

\CommentTok{# Row names are Gene|Uniprot}
\NormalTok{row_names <-}\StringTok{ }\KeywordTok{paste}\NormalTok{(gene_names, uniprot, }\DataTypeTok{sep =} \StringTok{"|"}\NormalTok{)}
\KeywordTok{head}\NormalTok{(row_names)}

\CommentTok{# Gather data, set row names}
\NormalTok{cols <-}\StringTok{ }\KeywordTok{grep}\NormalTok{(}\StringTok{"Abundance"}\NormalTok{, }\KeywordTok{colnames}\NormalTok{(data_in))}
\NormalTok{data_out <-}\StringTok{ }\KeywordTok{as.matrix}\NormalTok{(data_in[, cols])}
\KeywordTok{rownames}\NormalTok{(data_out) <-}\StringTok{ }\NormalTok{row_names}

\CommentTok{# Column names are batch.channel}
\NormalTok{idx <-}\StringTok{ }\KeywordTok{match}\NormalTok{(}\KeywordTok{colnames}\NormalTok{(data_out), sample_info}\OperatorTok{$}\NormalTok{ColumnName)}
\NormalTok{idx}
\NormalTok{col_names <-}\StringTok{ }\NormalTok{sample_info}\OperatorTok{$}\NormalTok{SampleID[idx]}
\KeywordTok{colnames}\NormalTok{(data_out) <-}\StringTok{ }\NormalTok{col_names}
\KeywordTok{head}\NormalTok{(col_names)}

\CommentTok{# Reorder based on batch.}
\NormalTok{data_out <-}\StringTok{ }\NormalTok{data_out[, }\KeywordTok{order}\NormalTok{(}\KeywordTok{colnames}\NormalTok{(data_out))]}

\CommentTok{# Save as cleanDat}
\NormalTok{cleanDat <-}\StringTok{ }\NormalTok{data_out}
\NormalTok{cleanDat[}\DecValTok{1}\OperatorTok{:}\DecValTok{5}\NormalTok{, }\DecValTok{1}\OperatorTok{:}\DecValTok{5}\NormalTok{]}
\KeywordTok{dim}\NormalTok{(cleanDat)}

\CommentTok{# Save to Rdata}
\NormalTok{file <-}\StringTok{ }\KeywordTok{paste}\NormalTok{(Rdatadir, tissue, }\StringTok{"CleanDat_IRS_eBLM_TAMPOR_format.Rds"}\NormalTok{, }\DataTypeTok{sep =} \StringTok{"/"}\NormalTok{)}
\KeywordTok{saveRDS}\NormalTok{(cleanDat, file)}
\CommentTok{# cleanDat <- readRDS(file)}

\CommentTok{# The cortex and striatum data from this chunk will be combined for TAMPOR}
\CommentTok{# normalization and statistical testing in the}
\CommentTok{# TMT_Analysis_EBD_v14_Combined_Cortex_Striatum.R script.}
\end{Highlighting}
\end{Shaded}

\subsection{InterBatch Statistical comparisons with EdgeR
GLM.}\label{interbatch-statistical-comparisons-with-edger-glm.}

After regression of genetic strain, pooling of WT samples will increase
statistical power.

\begin{Shaded}
\begin{Highlighting}[]
\CommentTok{# Create DGEList object with EB adjusted data.}
\NormalTok{data <-}\StringTok{ }\DecValTok{2}\OperatorTok{^}\KeywordTok{t}\NormalTok{(data.fit)}
\NormalTok{data[}\DecValTok{1}\OperatorTok{:}\DecValTok{5}\NormalTok{, }\DecValTok{1}\OperatorTok{:}\DecValTok{5}\NormalTok{]}
\NormalTok{y_DGE <-}\StringTok{ }\KeywordTok{DGEList}\NormalTok{(}\DataTypeTok{counts =}\NormalTok{ data)}

\CommentTok{# Example, checking the the normalization with plotMD.}
\KeywordTok{plotMD}\NormalTok{(}\KeywordTok{cpm}\NormalTok{(y_DGE, }\DataTypeTok{log =} \OtherTok{TRUE}\NormalTok{), }\DataTypeTok{column =} \DecValTok{2}\NormalTok{)}
\KeywordTok{abline}\NormalTok{(}\DataTypeTok{h =} \DecValTok{0}\NormalTok{, }\DataTypeTok{col =} \StringTok{"red"}\NormalTok{, }\DataTypeTok{lty =} \DecValTok{2}\NormalTok{, }\DataTypeTok{lwd =} \DecValTok{2}\NormalTok{)}

\CommentTok{# Create sample mapping.}
\NormalTok{traits <-}\StringTok{ }\NormalTok{sample_info}
\KeywordTok{rownames}\NormalTok{(traits) <-}\StringTok{ }\NormalTok{traits}\OperatorTok{$}\NormalTok{ColumnName}
\NormalTok{traits <-}\StringTok{ }\KeywordTok{subset}\NormalTok{(traits, }\KeywordTok{rownames}\NormalTok{(traits) }\OperatorTok\StringTok{ }\KeywordTok{colnames}\NormalTok{(data))}
\KeywordTok{all}\NormalTok{(traits}\OperatorTok{$}\NormalTok{ColumnName }\OperatorTok{==}\StringTok{ }\KeywordTok{colnames}\NormalTok{(data))}
\NormalTok{group <-}\StringTok{ }\NormalTok{traits}\OperatorTok{$}\NormalTok{Sample.Model}
\NormalTok{strain <-}\StringTok{ }\NormalTok{traits}\OperatorTok{$}\NormalTok{Model}
\NormalTok{sex <-}\StringTok{ }\NormalTok{traits}\OperatorTok{$}\NormalTok{Sex}
\KeywordTok{unique}\NormalTok{(group)}
\NormalTok{group[}\KeywordTok{grepl}\NormalTok{(}\StringTok{"WT"}\NormalTok{, group)] <-}\StringTok{ "WT"}
\KeywordTok{unique}\NormalTok{(group)}
\NormalTok{y_DGE}\OperatorTok{$}\NormalTok{samples}\OperatorTok{$}\NormalTok{group <-}\StringTok{ }\KeywordTok{as.factor}\NormalTok{(group)}

\CommentTok{# Basic design matrix for GLM.}
\NormalTok{design <-}\StringTok{ }\KeywordTok{model.matrix}\NormalTok{(}\OperatorTok{~}\StringTok{ }\DecValTok{0} \OperatorTok{+}\StringTok{ }\NormalTok{group, }\DataTypeTok{data =}\NormalTok{ y_DGE}\OperatorTok{$}\NormalTok{samples)}
\KeywordTok{colnames}\NormalTok{(design) <-}\StringTok{ }\KeywordTok{levels}\NormalTok{(y_DGE}\OperatorTok{$}\NormalTok{samples}\OperatorTok{$}\NormalTok{group)}
\NormalTok{design}

\CommentTok{# Estimate dispersion:}
\NormalTok{y_DGE <-}\StringTok{ }\KeywordTok{estimateDisp}\NormalTok{(y_DGE, design, }\DataTypeTok{robust =} \OtherTok{TRUE}\NormalTok{)}

\CommentTok{# PlotBCV}
\NormalTok{plot <-}\StringTok{ }\KeywordTok{ggplotBCV}\NormalTok{(y_DGE)}
\NormalTok{plot}

\CommentTok{# Fit a general linear model.}
\NormalTok{fit <-}\StringTok{ }\KeywordTok{glmQLFit}\NormalTok{(y_DGE, design, }\DataTypeTok{robust =} \OtherTok{TRUE}\NormalTok{)}

\CommentTok{# Examine the QL fitted dispersion.}
\KeywordTok{plotQLDisp}\NormalTok{(fit)}

\CommentTok{# Which genes are DE among all contrasts?}
\CommentTok{# Create contrast matrix for ANOVA-like comparison.}
\NormalTok{aov_contrasts <-}\StringTok{ }\KeywordTok{makeContrasts}\NormalTok{(}
  \DataTypeTok{WTvKO.Shank2 =}\NormalTok{ KO.Shank2 }\OperatorTok{-}\StringTok{ }\NormalTok{WT,}
  \DataTypeTok{WTvKO.Shank3 =}\NormalTok{ KO.Shank3 }\OperatorTok{-}\StringTok{ }\NormalTok{WT,}
  \DataTypeTok{WTvHET.Syngap1 =}\NormalTok{ HET.Syngap1 }\OperatorTok{-}\StringTok{ }\NormalTok{WT,}
  \DataTypeTok{WTvKO.Ube3a =}\NormalTok{ KO.Ube3a }\OperatorTok{-}\StringTok{ }\NormalTok{WT, }\DataTypeTok{levels =}\NormalTok{ design}
\NormalTok{)}
\NormalTok{aov_contrasts}

\CommentTok{# Calculate ANOVA-like results.}
\CommentTok{# The QL F-test is applied to identify genes that are DE among all four groups. This}
\CommentTok{# combines the four pairwise comparisons into a single F-statistic and p-value. The top set of}
\CommentTok{# significant genes can be displayed with topTags.}
\NormalTok{aov <-}\StringTok{ }\KeywordTok{glmQLFTest}\NormalTok{(fit, }\DataTypeTok{contrast =}\NormalTok{ aov_contrasts)}
\NormalTok{aov_tt <-}\StringTok{ }\KeywordTok{topTags}\NormalTok{(aov, }\DataTypeTok{n =} \OtherTok{Inf}\NormalTok{, }\DataTypeTok{sort.by =} \StringTok{"none"}\NormalTok{)}

\CommentTok{# Extract the results, and annotated with Entrez IDS and gene symbols.}
\NormalTok{res <-}\StringTok{ }\KeywordTok{annotate_Entrez}\NormalTok{(aov_tt}\OperatorTok{$}\NormalTok{table)}
\CommentTok{# Discard the CPM column.}
\NormalTok{res}\OperatorTok{$}\NormalTok{logCPM <-}\StringTok{ }\OtherTok{NULL}
\NormalTok{results_GLMoverall <-}\StringTok{ }\NormalTok{res}

\CommentTok{# Create a list of contrasts for pairwise comparisons.}
\NormalTok{contrasts <-}\StringTok{ }\KeywordTok{list}\NormalTok{(}
\NormalTok{  WTvKO.Shank2 <-}\StringTok{ }\KeywordTok{makeContrasts}\NormalTok{(KO.Shank2 }\OperatorTok{-}\StringTok{ }\NormalTok{WT, }\DataTypeTok{levels =}\NormalTok{ design),}
\NormalTok{  WTvKO.Shank3 <-}\StringTok{ }\KeywordTok{makeContrasts}\NormalTok{(KO.Shank3 }\OperatorTok{-}\StringTok{ }\NormalTok{WT, }\DataTypeTok{levels =}\NormalTok{ design),}
\NormalTok{  WTvHET.Syngap1 <-}\StringTok{ }\KeywordTok{makeContrasts}\NormalTok{(HET.Syngap1 }\OperatorTok{-}\StringTok{ }\NormalTok{WT, }\DataTypeTok{levels =}\NormalTok{ design),}
\NormalTok{  WTvKO.Ube3a <-}\StringTok{ }\KeywordTok{makeContrasts}\NormalTok{(KO.Ube3a }\OperatorTok{-}\StringTok{ }\NormalTok{WT, }\DataTypeTok{levels =}\NormalTok{ design)}
\NormalTok{)}

\CommentTok{# Call glmQLFTest() to evaluate differences in contrasts.}
\NormalTok{qlf <-}\StringTok{ }\KeywordTok{lapply}\NormalTok{(contrasts, }\ControlFlowTok{function}\NormalTok{(x) }\KeywordTok{glmQLFTest}\NormalTok{(fit, }\DataTypeTok{contrast =}\NormalTok{ x))}

\NormalTok{## Determine number of significant results with decideTests().}
\NormalTok{summary_table <-}\StringTok{ }\KeywordTok{lapply}\NormalTok{(qlf, }\ControlFlowTok{function}\NormalTok{(x) }\KeywordTok{summary}\NormalTok{(}\KeywordTok{decideTests}\NormalTok{(x)))}
\NormalTok{overall <-}\StringTok{ }\KeywordTok{t}\NormalTok{(}\KeywordTok{matrix}\NormalTok{(}\KeywordTok{unlist}\NormalTok{(summary_table), }\DataTypeTok{nrow =} \DecValTok{3}\NormalTok{, }\DataTypeTok{ncol =} \DecValTok{4}\NormalTok{))}
\KeywordTok{rownames}\NormalTok{(overall) <-}\StringTok{ }\KeywordTok{unlist}\NormalTok{(}\KeywordTok{lapply}\NormalTok{(contrasts, }\ControlFlowTok{function}\NormalTok{(x) }\KeywordTok{colnames}\NormalTok{(x)))}
\KeywordTok{colnames}\NormalTok{(overall) <-}\StringTok{ }\KeywordTok{c}\NormalTok{(}\StringTok{"Down"}\NormalTok{, }\StringTok{"NotSig"}\NormalTok{, }\StringTok{"Up"}\NormalTok{)}
\NormalTok{overall <-}\StringTok{ }\KeywordTok{as.data.frame}\NormalTok{(overall)}
\NormalTok{overall <-}\StringTok{ }\KeywordTok{add_column}\NormalTok{(overall, }\DataTypeTok{Contrast =} \KeywordTok{rownames}\NormalTok{(overall), }\DataTypeTok{.before =} \DecValTok{1}\NormalTok{)}
\NormalTok{overall <-}\StringTok{ }\NormalTok{overall[, }\KeywordTok{c}\NormalTok{(}\DecValTok{1}\NormalTok{, }\DecValTok{3}\NormalTok{, }\DecValTok{2}\NormalTok{, }\DecValTok{4}\NormalTok{)]}
\NormalTok{overall}\OperatorTok{$}\NormalTok{TotalSig <-}\StringTok{ }\KeywordTok{rowSums}\NormalTok{(overall[, }\KeywordTok{c}\NormalTok{(}\DecValTok{3}\NormalTok{, }\DecValTok{4}\NormalTok{)])}
\CommentTok{# Table of DE candidates.}
\NormalTok{table <-}\StringTok{ }\KeywordTok{tableGrob}\NormalTok{(overall, }\DataTypeTok{rows =} \OtherTok{NULL}\NormalTok{)}
\KeywordTok{grid.arrange}\NormalTok{(table)}

\CommentTok{# Save table.}
\NormalTok{file <-}\StringTok{ }\KeywordTok{paste0}\NormalTok{(outputfigsdir, }\StringTok{"/"}\NormalTok{, outputMatName, }\StringTok{"_InterBatch_eBLM_Table.tiff"}\NormalTok{)}
\KeywordTok{ggsave}\NormalTok{(file,table)}

\CommentTok{# Call topTags to add FDR. Gather tablurized results.}
\NormalTok{results <-}\StringTok{ }\KeywordTok{lapply}\NormalTok{(qlf, }\ControlFlowTok{function}\NormalTok{(x) }\KeywordTok{topTags}\NormalTok{(x, }\DataTypeTok{n =} \OtherTok{Inf}\NormalTok{, }\DataTypeTok{sort.by =} \StringTok{"none"}\NormalTok{)}\OperatorTok{$}\NormalTok{table)}

\CommentTok{# Function to annotate DE candidates:}
\NormalTok{annotateTopTags <-}\StringTok{ }\ControlFlowTok{function}\NormalTok{(y_TT) \{}
\NormalTok{  y_TT}\OperatorTok{$}\NormalTok{logCPM <-}\StringTok{ }\DecValTok{100} \OperatorTok{*}\StringTok{ }\NormalTok{(}\DecValTok{2}\OperatorTok{^}\NormalTok{y_TT}\OperatorTok{$}\NormalTok{logFC)}
  \KeywordTok{colnames}\NormalTok{(y_TT)[}\DecValTok{2}\NormalTok{] <-}\StringTok{ "%WT"}
  \KeywordTok{colnames}\NormalTok{(y_TT)[}\DecValTok{3}\NormalTok{] <-}\StringTok{ "F Value"}
\NormalTok{  y_TT}\OperatorTok{$}\NormalTok{candidate <-}\StringTok{ "no"}
\NormalTok{  y_TT[}\KeywordTok{which}\NormalTok{(y_TT}\OperatorTok{$}\NormalTok{FDR }\OperatorTok{<=}\StringTok{ }\FloatTok{0.10} \OperatorTok{&}\StringTok{ }\NormalTok{y_TT}\OperatorTok{$}\NormalTok{FDR }\OperatorTok{>}\StringTok{ }\FloatTok{0.05}\NormalTok{), }\KeywordTok{dim}\NormalTok{(y_TT)[}\DecValTok{2}\NormalTok{]] <-}\StringTok{ "low"}
\NormalTok{  y_TT[}\KeywordTok{which}\NormalTok{(y_TT}\OperatorTok{$}\NormalTok{FDR }\OperatorTok{<=}\StringTok{ }\FloatTok{0.05} \OperatorTok{&}\StringTok{ }\NormalTok{y_TT}\OperatorTok{$}\NormalTok{FDR }\OperatorTok{>}\StringTok{ }\FloatTok{0.01}\NormalTok{), }\KeywordTok{dim}\NormalTok{(y_TT)[}\DecValTok{2}\NormalTok{]] <-}\StringTok{ "med"}
\NormalTok{  y_TT[}\KeywordTok{which}\NormalTok{(y_TT}\OperatorTok{$}\NormalTok{FDR }\OperatorTok{<=}\StringTok{ }\FloatTok{0.01}\NormalTok{), }\KeywordTok{dim}\NormalTok{(y_TT)[}\DecValTok{2}\NormalTok{]] <-}\StringTok{ "high"}
\NormalTok{  y_TT}\OperatorTok{$}\NormalTok{candidate <-}\StringTok{ }\KeywordTok{factor}\NormalTok{(y_TT}\OperatorTok{$}\NormalTok{candidate, }\DataTypeTok{levels =} \KeywordTok{c}\NormalTok{(}\StringTok{"high"}\NormalTok{, }\StringTok{"med"}\NormalTok{, }\StringTok{"low"}\NormalTok{, }\StringTok{"no"}\NormalTok{))}
  \KeywordTok{return}\NormalTok{(y_TT)}
\NormalTok{\}}

\CommentTok{# Convert logCPM column to percent WT.}
\CommentTok{# Annotate with candidate column.}
\NormalTok{results <-}\StringTok{ }\KeywordTok{lapply}\NormalTok{(results, }\ControlFlowTok{function}\NormalTok{(x) }\KeywordTok{annotateTopTags}\NormalTok{(x))}

\CommentTok{# Annotate with Gene names and Entrez IDS.}
\NormalTok{results <-}\StringTok{ }\KeywordTok{lapply}\NormalTok{(results, }\ControlFlowTok{function}\NormalTok{(x) }\KeywordTok{annotate_Entrez}\NormalTok{(x))}

\NormalTok{## Add GLM overall stats and fitted data to results.}
\NormalTok{GLM_stats <-}\StringTok{ }\NormalTok{results_GLMoverall[, }\KeywordTok{c}\NormalTok{(}\DecValTok{8}\NormalTok{, }\DecValTok{9}\NormalTok{, }\DecValTok{10}\NormalTok{)]}
\KeywordTok{colnames}\NormalTok{(GLM_stats)[}\DecValTok{1}\NormalTok{] <-}\StringTok{ "F Value"}
\KeywordTok{colnames}\NormalTok{(GLM_stats) <-}\StringTok{ }\KeywordTok{paste}\NormalTok{(}\StringTok{"Overall"}\NormalTok{, }\KeywordTok{colnames}\NormalTok{(GLM_stats))}
\KeywordTok{names}\NormalTok{(results) <-}\StringTok{ }\NormalTok{groups}
\NormalTok{data_glm <-}\StringTok{ }\KeywordTok{log2}\NormalTok{(qlf[[}\DecValTok{1}\NormalTok{]]}\OperatorTok{$}\NormalTok{fitted.values)}
\NormalTok{data_glm <-}\StringTok{ }\KeywordTok{merge}\NormalTok{(GLM_stats, data_glm, }\DataTypeTok{by =} \StringTok{"row.names"}\NormalTok{, }\DataTypeTok{all =} \OtherTok{TRUE}\NormalTok{)}
\KeywordTok{rownames}\NormalTok{(data_glm) <-}\StringTok{ }\NormalTok{data_glm}\OperatorTok{$}\NormalTok{Row.names}
\NormalTok{data_glm}\OperatorTok{$}\NormalTok{Row.names <-}\StringTok{ }\OtherTok{NULL}
\NormalTok{data_glm[}\DecValTok{1}\OperatorTok{:}\DecValTok{5}\NormalTok{, }\DecValTok{1}\OperatorTok{:}\DecValTok{5}\NormalTok{]}

\CommentTok{# Loop to add fitted data + GLM overall stats to pairwise comparisons in results.}
\ControlFlowTok{for}\NormalTok{ (i }\ControlFlowTok{in} \DecValTok{1}\OperatorTok{:}\KeywordTok{length}\NormalTok{(groups)) \{}
\NormalTok{  cols <-}\StringTok{ }\KeywordTok{c}\NormalTok{(}\DecValTok{1}\NormalTok{, }\DecValTok{2}\NormalTok{, }\DecValTok{3}\NormalTok{, }\KeywordTok{grep}\NormalTok{(}\KeywordTok{paste}\NormalTok{(groups[i], }\StringTok{"WT"}\NormalTok{, }\DataTypeTok{sep =} \StringTok{"|"}\NormalTok{), }\KeywordTok{colnames}\NormalTok{(data_glm)))}
\NormalTok{  results[[i]] <-}\StringTok{ }\KeywordTok{merge}\NormalTok{(results[[i]], data_glm[, cols], }\DataTypeTok{by =} \StringTok{"row.names"}\NormalTok{)}
\NormalTok{  results[[i]]}\OperatorTok{$}\NormalTok{Row.names <-}\StringTok{ }\OtherTok{NULL}
\NormalTok{  results[[i]] <-}\StringTok{ }\NormalTok{results[[i]][}\KeywordTok{order}\NormalTok{(results[[i]]}\OperatorTok{$}\NormalTok{PValue), ]}
\NormalTok{\}}

\CommentTok{# Loop to Re-organize columns.}
\ControlFlowTok{for}\NormalTok{ (i }\ControlFlowTok{in} \DecValTok{1}\OperatorTok{:}\KeywordTok{length}\NormalTok{(groups)) \{}
\NormalTok{  col_names <-}\StringTok{ }\KeywordTok{colnames}\NormalTok{(results[[i]])}
\NormalTok{  colsWT <-}\StringTok{ }\KeywordTok{grep}\NormalTok{(}\StringTok{"WT"}\NormalTok{, col_names)[}\OperatorTok{-}\DecValTok{1}\NormalTok{]}
\NormalTok{  colsKO <-}\StringTok{ }\KeywordTok{grep}\NormalTok{(}\StringTok{"KO|HET"}\NormalTok{, col_names)}
\NormalTok{  colsOverall <-}\StringTok{ }\KeywordTok{grep}\NormalTok{(}\StringTok{"Overall"}\NormalTok{, col_names)}
\NormalTok{  colsElse <-}\StringTok{ }\NormalTok{(}\DecValTok{1}\OperatorTok{:}\KeywordTok{length}\NormalTok{(col_names))[}\OperatorTok{-}\KeywordTok{c}\NormalTok{(}\DecValTok{1}\NormalTok{, }\DecValTok{2}\NormalTok{, }\DecValTok{3}\NormalTok{, colsWT, colsKO, colsOverall)]}
\NormalTok{  idx <-}\StringTok{ }\KeywordTok{c}\NormalTok{(}\DecValTok{1}\NormalTok{, }\DecValTok{2}\NormalTok{, }\DecValTok{3}\NormalTok{, colsOverall, colsElse, colsKO, colsWT)}
\NormalTok{  col_names[idx]}
\NormalTok{  results[[i]] <-}\StringTok{ }\NormalTok{results[[i]][col_names[idx]]}
\NormalTok{\}}

\CommentTok{# Results}
\NormalTok{results_interBatch <-}\StringTok{ }\NormalTok{results}

\CommentTok{# Pvalue Histograms (need to ingore "P overall" column):}
\NormalTok{p1 <-}\StringTok{ }\KeywordTok{ggplotPvalHist}\NormalTok{(results[[}\DecValTok{1}\NormalTok{]][}\OperatorTok{-}\KeywordTok{c}\NormalTok{(}\DecValTok{1}\OperatorTok{:}\DecValTok{6}\NormalTok{)], }\StringTok{"gold1"}\NormalTok{, }\StringTok{"Shank2"}\NormalTok{)}
\NormalTok{p2 <-}\StringTok{ }\KeywordTok{ggplotPvalHist}\NormalTok{(results[[}\DecValTok{2}\NormalTok{]][}\OperatorTok{-}\KeywordTok{c}\NormalTok{(}\DecValTok{1}\OperatorTok{:}\DecValTok{6}\NormalTok{)], }\StringTok{"blue"}\NormalTok{, }\StringTok{"Shank3"}\NormalTok{)}
\NormalTok{p3 <-}\StringTok{ }\KeywordTok{ggplotPvalHist}\NormalTok{(results[[}\DecValTok{3}\NormalTok{]][}\OperatorTok{-}\KeywordTok{c}\NormalTok{(}\DecValTok{1}\OperatorTok{:}\DecValTok{6}\NormalTok{)], }\StringTok{"green"}\NormalTok{, }\StringTok{"Syngap1"}\NormalTok{)}
\NormalTok{p4 <-}\StringTok{ }\KeywordTok{ggplotPvalHist}\NormalTok{(results[[}\DecValTok{4}\NormalTok{]][}\OperatorTok{-}\KeywordTok{c}\NormalTok{(}\DecValTok{1}\OperatorTok{:}\DecValTok{6}\NormalTok{)], }\StringTok{"purple"}\NormalTok{, }\StringTok{"Ube3a"}\NormalTok{)}
\NormalTok{fig <-}\StringTok{ }\KeywordTok{plot_grid}\NormalTok{(p1, p2, p3, p4, }\DataTypeTok{labels =} \StringTok{"auto"}\NormalTok{)}
\NormalTok{fig}

\CommentTok{# Save plots.}
\CommentTok{#file <- paste0(outputfigsdir, "/", outputMatName, "_InterBatch_GLM_PvalHist.pdf")}
\CommentTok{#ggsavePDF(plots = list(table, p1, p2, p3, p4), file)}

\CommentTok{# Save as tiff.}
\NormalTok{file <-}\StringTok{ }\KeywordTok{paste0}\NormalTok{(outputfigsdir, }\StringTok{"/"}\NormalTok{, outputMatName, }\StringTok{"_InterBatch_GLM_PvalHist.tiff"}\NormalTok{)}
\KeywordTok{ggsave}\NormalTok{(file,fig)}

\NormalTok{## Write results to file.}
\CommentTok{# Add summary table to results.}
\NormalTok{overall <-}\StringTok{ }\KeywordTok{list}\NormalTok{(overall)}
\KeywordTok{names}\NormalTok{(overall) <-}\StringTok{ "Summary"}
\NormalTok{results <-}\StringTok{ }\KeywordTok{c}\NormalTok{(overall, results)}

\CommentTok{# Save workbook.}
\NormalTok{file <-}\StringTok{ }\KeywordTok{paste0}\NormalTok{(outputtabsdir, }\StringTok{"/"}\NormalTok{, outputMatName, }\StringTok{"_InterBatch_eBLM_GLM_Results.xlsx"}\NormalTok{)}
\KeywordTok{write.excel}\NormalTok{(results, file)}

\CommentTok{# Save to RDS.}
\NormalTok{file <-}\StringTok{ }\KeywordTok{paste0}\NormalTok{(Rdatadir, }\StringTok{"/"}\NormalTok{, outputMatName, }\StringTok{"_InterBatch_Results.RDS"}\NormalTok{)}
\KeywordTok{saveRDS}\NormalTok{(results_interBatch, file)}
\end{Highlighting}
\end{Shaded}

\subsection{InterBatch GO and KEGG enrichment
testing.}\label{interbatch-go-and-kegg-enrichment-testing.}

\begin{Shaded}
\begin{Highlighting}[]
\NormalTok{## Perform GO and KEGG testing.}
\CommentTok{# qlf_GSE() is a wrapper around the goana() and kegga() functions from EdgeR.}
\CommentTok{# This function operates on the QLF object.}
\CommentTok{# Proteins with FDR <0.05 will be considered differentially expressed.}

\CommentTok{# Use lapply to generate GSE results with custom function edgeR_GSE().}
\CommentTok{# This will take a few minutes.}
\NormalTok{GSE_results <-}\StringTok{ }\KeywordTok{lapply}\NormalTok{(qlf, }\ControlFlowTok{function}\NormalTok{(x) }\KeywordTok{edgeR_GSE}\NormalTok{(x, }\DataTypeTok{FDR =} \FloatTok{0.05}\NormalTok{, }\DataTypeTok{filter =} \OtherTok{TRUE}\NormalTok{))}

\CommentTok{# Name GSE_results.}
\NormalTok{GSE_results <-}\StringTok{ }\KeywordTok{unlist}\NormalTok{(GSE_results, }\DataTypeTok{recursive =} \OtherTok{FALSE}\NormalTok{)}
\KeywordTok{names}\NormalTok{(GSE_results) <-}\StringTok{ }\KeywordTok{paste}\NormalTok{(}\KeywordTok{rep}\NormalTok{(groups, }\DataTypeTok{each =} \DecValTok{2}\NormalTok{), }\KeywordTok{names}\NormalTok{(GSE_results))}

\CommentTok{# Initiate an excel workbook.}
\NormalTok{wb <-}\StringTok{ }\KeywordTok{createWorkbook}\NormalTok{()}

\CommentTok{# Loop to add a worksheets:}
\ControlFlowTok{for}\NormalTok{ (i }\ControlFlowTok{in} \DecValTok{1}\OperatorTok{:}\KeywordTok{length}\NormalTok{(GSE_results)) \{}
\NormalTok{  df <-}\StringTok{ }\NormalTok{GSE_results[[i]]}
  \KeywordTok{addWorksheet}\NormalTok{(wb, }\DataTypeTok{sheetName =} \KeywordTok{names}\NormalTok{(GSE_results[i]))}
  \KeywordTok{writeData}\NormalTok{(wb, }\DataTypeTok{sheet =}\NormalTok{ i, df)}
\NormalTok{\}}

\CommentTok{# Save workbook.}
\NormalTok{file <-}\StringTok{ }\KeywordTok{paste0}\NormalTok{(outputtabsdir, }\StringTok{"/"}\NormalTok{, outputMatName, }\StringTok{"_InterBatch_eBLM_GLM_GSE_Results.xlsx"}\NormalTok{)}
\KeywordTok{saveWorkbook}\NormalTok{(wb, file, }\DataTypeTok{overwrite =} \OtherTok{TRUE}\NormalTok{)}
\end{Highlighting}
\end{Shaded}

\subsection{Generate protein boxplots for significantly DE proteins
(Interbatch).}\label{generate-protein-boxplots-for-significantly-de-proteins-interbatch.}

\begin{Shaded}
\begin{Highlighting}[]
\CommentTok{# Subset the data. Keep proteins with p.adj overall <0.05.}
\NormalTok{data_sub <-}\StringTok{ }\NormalTok{results}\OperatorTok{$}\NormalTok{Shank2[results_interBatch}\OperatorTok{$}\NormalTok{Shank2}\OperatorTok{$}\StringTok{`}\DataTypeTok{Overall PValue}\StringTok{`} \OperatorTok{<}\StringTok{ }\FloatTok{0.05}\NormalTok{, ]}
\NormalTok{keep <-}\StringTok{ }\NormalTok{data_sub}\OperatorTok{$}\NormalTok{Uniprot}
\NormalTok{exprDat <-}\StringTok{ }\KeywordTok{log2}\NormalTok{(y_DGE}\OperatorTok{$}\NormalTok{counts)}
\NormalTok{idx <-}\StringTok{ }\KeywordTok{rownames}\NormalTok{(exprDat) }\OperatorTok\StringTok{ }\NormalTok{keep}
\NormalTok{exprDat <-}\StringTok{ }\NormalTok{exprDat[idx, ]}

\CommentTok{# Annotate exprDat rows as gene|uniprot}
\NormalTok{exprDat <-}\StringTok{ }\KeywordTok{log2}\NormalTok{(y_DGE}\OperatorTok{$}\NormalTok{counts)}
\NormalTok{Uniprot <-}\StringTok{ }\KeywordTok{rownames}\NormalTok{(exprDat)}
\NormalTok{Gene <-}\StringTok{ }\KeywordTok{mapIds}\NormalTok{(}
\NormalTok{  org.Mm.eg.db,}
  \DataTypeTok{keys =}\NormalTok{ Uniprot,}
  \DataTypeTok{column =} \StringTok{"SYMBOL"}\NormalTok{,}
  \DataTypeTok{keytype =} \StringTok{"UNIPROT"}\NormalTok{,}
  \DataTypeTok{multiVals =} \StringTok{"first"}
\NormalTok{)}
\KeywordTok{rownames}\NormalTok{(exprDat) <-}\StringTok{ }\KeywordTok{paste}\NormalTok{(Gene, }\KeywordTok{rownames}\NormalTok{(exprDat), }\DataTypeTok{sep =} \StringTok{"|"}\NormalTok{)}

\CommentTok{# Insure all WT samples are annotated as WT in traits.}
\NormalTok{traits_temp <-}\StringTok{ }\NormalTok{traits}
\NormalTok{traits_temp}\OperatorTok{$}\NormalTok{Sample.Model[}\KeywordTok{grepl}\NormalTok{(}\StringTok{"WT"}\NormalTok{, traits_temp}\OperatorTok{$}\NormalTok{Sample.Model)] <-}\StringTok{ "WT"}
\NormalTok{traits_temp <-}\StringTok{ }\KeywordTok{subset}\NormalTok{(traits_temp, }\KeywordTok{rownames}\NormalTok{(traits_temp) }\OperatorTok\StringTok{ }\KeywordTok{colnames}\NormalTok{(exprDat))}

\CommentTok{# Generate plots.}
\NormalTok{plot_list <-}\StringTok{ }\KeywordTok{ggplotProteinBoxes}\NormalTok{(}
  \DataTypeTok{data_in =}\NormalTok{ exprDat,}
  \DataTypeTok{interesting.proteins =} \KeywordTok{rownames}\NormalTok{(exprDat),}
  \DataTypeTok{dataType =} \StringTok{"Relative Abundance"}\NormalTok{,}
  \DataTypeTok{traits =}\NormalTok{ traits_temp,}
  \DataTypeTok{order =} \KeywordTok{c}\NormalTok{(}\DecValTok{1}\NormalTok{, }\DecValTok{2}\NormalTok{, }\DecValTok{3}\NormalTok{, }\DecValTok{4}\NormalTok{, }\DecValTok{5}\NormalTok{),}
  \DataTypeTok{scatter =} \OtherTok{TRUE}
\NormalTok{)}
\CommentTok{# Add custom colors.}
\NormalTok{colors <-}\StringTok{ }\KeywordTok{c}\NormalTok{(}\StringTok{"gray"}\NormalTok{, }\StringTok{"yellow"}\NormalTok{, }\StringTok{"blue"}\NormalTok{, }\StringTok{"green"}\NormalTok{, }\StringTok{"purple"}\NormalTok{)}
\NormalTok{plot_list <-}\StringTok{ }\KeywordTok{lapply}\NormalTok{(plot_list, }\ControlFlowTok{function}\NormalTok{(x) x }\OperatorTok{+}\StringTok{ }\KeywordTok{scale_fill_manual}\NormalTok{(}\DataTypeTok{values =}\NormalTok{ colors))}

\CommentTok{# Example plot.}
\NormalTok{plot_list[[}\DecValTok{1}\NormalTok{]]}

\NormalTok{## Add significance stars.}
\CommentTok{# Build a df with statistical results.}
\NormalTok{stats <-}\StringTok{ }\KeywordTok{lapply}\NormalTok{(results_interBatch, }\ControlFlowTok{function}\NormalTok{(x)}
  \KeywordTok{as.data.frame}\NormalTok{(}\KeywordTok{cbind}\NormalTok{(}\DataTypeTok{Uniprot =}\NormalTok{ x}\OperatorTok{$}\NormalTok{Uniprot, }\DataTypeTok{FDR =}\NormalTok{ x}\OperatorTok{$}\NormalTok{FDR)))}
\KeywordTok{names}\NormalTok{(stats) <-}\StringTok{ }\KeywordTok{c}\NormalTok{(}\StringTok{"KO.Shank2"}\NormalTok{, }\StringTok{"KO.Shank3"}\NormalTok{, }\StringTok{"HET.Syngap1"}\NormalTok{, }\StringTok{"KO.Ube3a"}\NormalTok{)}
\NormalTok{df <-}\StringTok{ }\NormalTok{stats }\OperatorTok\StringTok{ }\KeywordTok{reduce}\NormalTok{(left_join, }\DataTypeTok{by =} \StringTok{"Uniprot"}\NormalTok{)}
\KeywordTok{colnames}\NormalTok{(df)[}\KeywordTok{c}\NormalTok{(}\DecValTok{2}\OperatorTok{:}\KeywordTok{ncol}\NormalTok{(df))] <-}\StringTok{ }\KeywordTok{names}\NormalTok{(stats)}

\CommentTok{# Annotate rows as gene|uniprot}
\NormalTok{Uniprot <-}\StringTok{ }\NormalTok{df}\OperatorTok{$}\NormalTok{Uniprot}
\NormalTok{Gene <-}\StringTok{ }\KeywordTok{mapIds}\NormalTok{(}
\NormalTok{  org.Mm.eg.db,}
  \DataTypeTok{keys =}\NormalTok{ Uniprot,}
  \DataTypeTok{column =} \StringTok{"SYMBOL"}\NormalTok{,}
  \DataTypeTok{keytype =} \StringTok{"UNIPROT"}\NormalTok{,}
  \DataTypeTok{multiVals =} \StringTok{"first"}
\NormalTok{)}
\KeywordTok{rownames}\NormalTok{(df) <-}\StringTok{ }\KeywordTok{paste}\NormalTok{(Gene, Uniprot, }\DataTypeTok{sep =} \StringTok{"|"}\NormalTok{)}
\NormalTok{df}\OperatorTok{$}\NormalTok{Uniprot <-}\StringTok{ }\OtherTok{NULL}
\NormalTok{stats <-}\StringTok{ }\NormalTok{df}

\CommentTok{# Example plot.}
\NormalTok{plot <-}\StringTok{ }\NormalTok{plot_list[[}\DecValTok{1}\NormalTok{]]}
\KeywordTok{annotate_stars}\NormalTok{(plot, stats)}

\CommentTok{# Loop to add stars.}
\NormalTok{plot_list <-}\StringTok{ }\KeywordTok{lapply}\NormalTok{(plot_list, }\ControlFlowTok{function}\NormalTok{(x) }\KeywordTok{annotate_stars}\NormalTok{(x, stats))}

\CommentTok{# Top proteins.}
\NormalTok{p1 <-}\StringTok{ }\NormalTok{plot_list}\OperatorTok{$}\StringTok{`}\DataTypeTok{Shank2|Q80Z38}\StringTok{`}
\NormalTok{p2 <-}\StringTok{ }\NormalTok{plot_list}\OperatorTok{$}\StringTok{`}\DataTypeTok{Shank3|Q4ACU6}\StringTok{`}
\NormalTok{p3 <-}\StringTok{ }\NormalTok{plot_list}\OperatorTok{$}\StringTok{`}\DataTypeTok{Syngap1|F6SEU4}\StringTok{`}
\NormalTok{p4 <-}\StringTok{ }\NormalTok{plot_list}\OperatorTok{$}\StringTok{`}\DataTypeTok{Ube3a|O08759}\StringTok{`}
\NormalTok{fig <-}\StringTok{ }\KeywordTok{plot_grid}\NormalTok{(p1,p2,p3,p4)}
\NormalTok{fig}

\CommentTok{# Save to pdf.}
\CommentTok{#file <- paste0(outputfigsdir, "/", tissue, "_WGCNA_Analysis_InterBatch_ProteinBoxPlots.pdf")}
\CommentTok{#ggsavePDF(plots = plot_list, file)}

\CommentTok{# Save as tiff.}
\NormalTok{file <-}\StringTok{ }\KeywordTok{paste0}\NormalTok{(outputfigsdir, }\StringTok{"/"}\NormalTok{, outputMatName, }\StringTok{"TopProteins_BoxPlots.tiff"}\NormalTok{)}
\KeywordTok{ggsave}\NormalTok{(file,fig)}
\end{Highlighting}
\end{Shaded}

\subsection{Render RMarkdown report.}\label{render-rmarkdown-report.}

This script is formatted for automated rendering of an RMarkdown report.

\begin{Shaded}
\begin{Highlighting}[]
\CommentTok{# Code directory. }
\NormalTok{dir <-}\StringTok{ }\KeywordTok{paste}\NormalTok{(rootdir,}\StringTok{"Code"}\NormalTok{,}\DataTypeTok{sep=}\StringTok{"/"}\NormalTok{)}
\NormalTok{file <-}\StringTok{ }\KeywordTok{paste}\NormalTok{(dir,}\StringTok{"1_TMT_Analysis.R"}\NormalTok{,}\DataTypeTok{sep=}\StringTok{"/"}\NormalTok{)}

\CommentTok{# Save and render.}
\NormalTok{rstudioapi}\OperatorTok{::}\KeywordTok{documentSave}\NormalTok{()}
\NormalTok{rmarkdown}\OperatorTok{::}\KeywordTok{render}\NormalTok{(file)}
\end{Highlighting}
\end{Shaded}


\end{document}
